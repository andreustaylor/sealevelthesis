\newpage
\subsection{Tides and data assimilation in operational meso-scale ocean prediction}
\label{S:plan_DA}

Following the conceptual path between tidal harmonics and operational oceanography, it is pertinent to address the implications for data assimilation (DA).\\
A critical discussion of issues and prospects in this area is proposed as a concluding chapter.\\


\subsubsection{Overview of a critical discussion}
Prediction systems in the \GODAE{} heritage have intentionally excluded ocean tides to focus on mesoscale ocean turbulence [Section \ref{S:operational_oceanography}].  This approach has largely been guided and enabled by the assimilation of sea level anomaly (SLA) observations from nadir-looking altimeters.  
However, the very existence of realtime SLA datastreams has been enabled by global tide solutions founded on the historical record of observations from the same satellite platforms.\\
Thus DA has played two distinct and complimentary roles:
\begin{itemize}
\item historical observations to optimise global atlases of stationary tides \citep{Egbert:1996vr};
\item realtime observations to constrain mesoscale \OGCM{} forecasts. 
\end{itemize}


\BoxBegin
Given the prospect of explicitly resolving tidal dynamics in an \OGCM{}, how should data assimilation be approached with regard to operational sea level forecasts? \\
\BoxEnd

The motivation is summarised as follows:
\begin{enumerate}
\item it is desirable to include ocean tides in the \OGCM{} (more physical representation of barotropic conversion and mixing dynamics - Section \ref{S:tides_ogcm});
\item the skill of mesoscale ocean forecasts is reliant on DA and the contemporary global observing array;
\item naively extending the present \OGCM{} data assimilation to ingest SSH does not appear viable (tidal phase errors are likely to be more powerful than the targeted nontidal signal); 
\item the goal of simultaneously resolving ocean tides \emph{and} constraining mesoscale circulation invites a high level exploration of limitations and options.
\end{enumerate}

%require a re-evaluation of data assimilation approach.
%\item Naively introduced, explicit tidal resolution in the model would not be compatible with data assimilation due to the relative power and high frequencies of this component of sea level. 

The existing DA configuration of \BL{} (BODAS EnOI)  will provide the launching point for this discussion.  From the outset, the role of resolving tidal dynamics will be considered a tool to improve the nontidal ocean state estimation. \\




For instance, the follow sketch outlines a prospective hybrid approach, persisting with the concept of decomposing the ocean state into tidal and nontidal.\\
DA could be facilitated by employing two parallel instances of the \OGCM{}: one with full dynamics and another with only tidal forcing.   Drawing on an analogy with the UK storm tide service \cite{Horsburg:2009ui}, the difference between the models could provide a self-consistent estimate of sea level anomaly.\\
Key to this concept is the reality that representation of tidal dynamics in an \OGCM{} will always be compromised compared to best available tidal atlases. 

For instance, at each analysis step:   
\begin{inparaenum}[(i)]
\item full background state from tide-resolving \OGCM{} stepped forward in time;
\item complimentary tidal background from a tide-only version of the same \OGCM{} stepped forward in time;
\item difference states to give nontidal state estimate;
\item increments to nontidal ocean state derived from existing DA software;
\item reconstitute the full ocean state by superposition of updated nontidal state with best estimate tide prediction.
\end{inparaenum}




Supporting the discussion of prospective approaches, this discussion will draw on relevant literature and address questions such as:
\begin{itemize}
\item Are fundamental scale limits imposed by the observation array? \\
\item Would realistic changes to the observation array dictate the choice of approach? 
\item Do model parameterisations and the role bathymetry influence the balance of considerations?
\item Can the respective strengths of tidal and nontidal DA be exploited in a hybrid approach? 
\end{itemize}













