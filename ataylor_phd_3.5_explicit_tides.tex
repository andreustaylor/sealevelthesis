\newpage
\subsection{Coastal sea level impact of explicit tides in a regional \OGCM{}}
\label{S:plan_OFAMR}
\subsubsection{Problem and motivation}
Existing simulations nested within \BL{} present an opportunity to address the prospects for explicit tidal resolution in an \OGCM{}.   Just beyond the \BOM{} operational environment, regional developmental \OGCM{}s explicitly resolving some tidal dynamics include \ER{} and \OFAMHR{}.   Whilst the focus of these developments has been neither coastal nor tidal skill per se, the availability of simulation datasets will be leveraged towards an initial appreciation of the the coastal sea level impact of including tides in an regional \OGCM{}.\\


\BoxBegin{}
Do these nominally tide-resolving \OGCM{}s present any benefits for coastal sea level forecasting compared to the existing aggregation configuration?
\BoxEnd{}

It is emphasised that the physical formulation and spatial discretisation of a regional \OGCM{} are distinct from the regional models employed for surge forecasting.   For instance, it is expected that the representation of barotropic tides will be relatively poor by standard measures. However given the discussion in Section \ref{S:tides_ogcm}, understanding the impact of the coarsely resolved tidal motions is significant.



\subsubsection{Data sources}
Existing developmental datasets will form the basis of this investigation. The spatial extent of the available simulations are expected to be restricted to the Australian East Coast.\\
As the simulations are produced by nesting within \BL{}, the operational system will provide the background fields for comparison.\\
Temporal bounds for the comparison will be influenced by the previous analysis of East Coast sea level outlined in Section \ref{S:plan_CTW}.


\subsubsection{Method outlook}
Primarily this investigation will compare simulation output to coastal tide gauges along Australia's East Coast.  However, in order to gain insight into these results, the wavenumber spectra of the surrounding region will also be addressed.   The methods employed will in part respond to recent publications related to global tide resolving HYCOM simulations \cite{Richman:2012bz}.\\



The perspective gained will inform subsequent more in-depth studies regarding explicit tides within an operational \OGCM{}; including those proposed in Sections \ref{S:plan_OTIS} and \ref{S:plan_bodyforcing}.




