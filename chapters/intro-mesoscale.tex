\section{Mesoscale operational oceanography}
\label{sec:mesoscaleOperational}
Tide prediction has effectively been part of routine operations for several decades; whether or not the term ``operational'' has been applied.
Starting in the early 2000s, mesoscale ocean prediction has become operational in a manner that lagged and imitated the path of \NWP \citep{Harper:kb} rather than tide prediction. 
Like modern weather forecasting, operational oceanography is similarly a big science enterprise \citep{Petersen:2012tr} in the sense that it fundamentally relies on the coordination of large organisations, global networks and capital. 
The size and nature of this type of prediction stands in stark contrast to the ``small'' well-established practices of conventional tide prediction introduced in section \ref{sec:tidesOverview}.
The international Global Ocean Data Assimilation Experiment \GODAE{} serves as a historical reference point for how mesoscale prediction came into operational agencies like the Bureau:
``\textit{the central idea of \GODAE{} - to demonstrate the feasibility and utility of real-time, global ocean forecasting - was based on the experiences of the meteorological community in \dots{} FGGE}'' \citep{Bell:2009uv}.
From this original feasibility demonstration, the enterprise of mesoscale ocean forecasting has continued to mature across many national weather agencies; and the Australian case is documented by \citep{10.1080/1755876x.2019.1685834} in reviewing over 15 years of operational services.


In contrast to conventional tide prediction, sea level of itself is not the target of mesoscale forecasting; let alone coastal sea level.   But sea surface elevation is an important state variable for which forecasts are produced.  What these forecasts actually represent is not self-evident.  The manner in which sea level and tides are handled in systems like \BL{} is worthy of exposition in order to underpin the subsequent chapters describing methods for combining and evaluating sea level derived from these operational forecasts. 
  
%-----------------%
\subsection{Ocean general circulation models}
Ocean General Circulation Models (\OGCM{}s) are now a key component in oceanographic forecasting, but are only rendered operationally useful by the use of a global observation network and data assimilation to routinely maintain alignment with reality.
How this representation of reality relates to coastal sea level is however not self-evident.



An \OGCM{} simulates the physical ocean state via time step integrating a mixed boundary-condition/initial-condition problem.
In contrast to the tidal LTI conception of the ocean described in section \ref{sec:LTI}, an \OGCM{} conceives the global ocean as a turbulent continuum fluid; and the time and length scales of this turbulence effectively have no lower bound. The simulation thus needs to represent information cascades between turbulent scales as well as account for unresolved processes with sub-grid scale (SGS) parameterisations.
The distinction between resolved and unresolved scales is paramount.  ``Sensitive dependence on initial conditions in this turbulent flow \dots{} severely limits predictive capabilities [and] motivates a formulation of averaged or mean field fluid equations'' \citep[Sec 2.5]{Griffies:2004vs}.
Furthermore, this raises broad questions of how to interpret the output of a numerical ocean simulation.
Griffies perceives an \OGCM{} as representing the averaged motion of an infinite ensemble of hypothetical oceans; the imagined spread being proportional to the scales of SGS parameterisation.
Alternatively, \citet{Stevens:2001kb} casts \OGCM{}s into the class of geophysical ``pseudofluid'' simulations. Of which he notes comparisons with observational data are prone to interpretational nuance and are ``too often ad hoc, uncritical, and/or irrelevant''\citep[pp 286]{Stevens:2001kb}. 
Figure \ref{fig:ogcmScales} illustrates very schematically the place of \OGCM{} variants in the context of turbulence simulations.   The exclusion or explicit representation of tidal dynamics are shown as changes in SGS complexity; these are discussed further below.
%----------------------------------
\begin{figure}[h]\centering
  \includegraphics[width=\figwidthBig]{figures/diagrams/ogcm_scales.pdf}
  \caption{Schematic place of \OGCM{}s in terms of sub-grid-scale parameterisations.}
  {Simulation of a turbulent ocean requires design choices regarding scales represented.  The schematic indicates the relative place of \OGCM{}s in contrast to simluations including Direct Numerical Simulation and Large Eddy Simulation. Following \citep[fig 5.2]{Petersen:2012tr} and \citep{Stevens:2001kb}.  Explicit tides in an OGCM represent a reduction of parameterisation without a large change in spatial scale resolution.}
  \label{fig:ogcmScales}
\end{figure}
%----------------------------------
Common to big science simulation practice more generally \citep{Petersen:2012tr} any real \OGCM{}, including \BL{}, embodies a significant accumulation of human effort in the very many lines of computer code.  In contrast to typical tidal analysis tools, the software behind a mesoscale ocean forecast is effectively too much for any one person to understand in detail.
%-----------------%
\subsection{Data assimilation and sea level observation}
Data assimilation is the model-data-fusion framework by which the initial conditions for each \OGCM{} forecast are determined.
This processing step can be usefully viewed from a Bayesian perspective in which ``the system implements an optimality criterion which defines how to best combine dynamics and observations, given an hypothesized error model for both.'' \citep{10.1007/978-94-007-0332-2_13}.
Subsequently the availability of near-real-time observations of the global ocean has played a critical role in enabling operational mesoscale forecasting.  Expansion of ocean observation coverage of recent decades largely mirrors the maturation of abundant satellite communications.  But despite the growth of what has been called the global ocean observation system \GOOS{} \citep{Komen:1999ch}, the ocean remains fundamentally under-observed for the purposes of physical state estimation.   Satellite-based surface observations of the ocean are the most abundant, whilst the ocean depths are sampled by comparatively sparse insitu profiles.


Tide gauges observations are fundamental to conventional tide prediction, but are in contrast often totally excluded from mesoscale data assimilation.   The fact that \BL{} does not assimilate any sea level data from tide gauges is relevant to the aggregated forecast design described in chapter\ref{chp:aggregate}.      
Sea level close to coast lines in a mesoscale ocean simulation is primarily an expression of the simulated fluid dynamics rather than an observationally constrained parameter.

Whilst tide gauges are excluded from assimilation, sea level observations derived from satellite mounted radar instruments are a crucial input and have played a pivotal role in the evolution of ocean forecasting in general \citep{Fu:2001ub}.  Moreover, the forecast skill associated with the assimilation of satellite altimeter data has been shown to be of particular significance by  \citep{10.5194/os-13-1077-2017} and others.
In contrast to the still water level observed by a tide gauge, it is primarily a sea level anomaly (SLA) quantity that is derived from the altimeters and used to constrain the \OGCM{}. Figure \ref{altimeterEg} illustrates the nature of the along-track nadir coverage of a single satellite altimeter.
These SLA observations are not however applied to constrain the ocean state in the vicinity of coastal tide gauges; and in the case of \BL{} are excluded from the assimilation process for regions where the depth is less than 200m or close to a land mass.   This reflects that fact that the derivation of sea level quantities from satellite altimeters is particularly challenging in shallow water and near coastal boundaries \citep{Woodworth:2011bf}.    Although an active area of development, coastal altimetry data streams are not yet used to constrain operational mesoscale forecast  systems.
%----------------------------------
\begin{figure}[h]\centering
  \includegraphics[width=\figwidthHalf]{figures/maps/altimeterCoverageEg.png}
  \caption{Typical spatial coverage of a satellite altimeter.}
          {The along-ground track accumulated over a single day by Saral in May 2020 is shown. Observations near the coast are excluded from assimilation.}
  \label{fig:altimeterEg}
\end{figure}
%----------------------------------
SLA is a meaningful quantity to the extent that it is an anomaly from some reference, and that reference involves a spatial tide prediction; as introduced above in section \ref{sec:spatialTides}.
The role of tides in mesoscale forecasts is taken up in the following sections.
%--------------------------------------------------
\subsection{Treatment of tides in nominally non-tidal \OGCM{}s}
\label{sec:tides_ogcm}
Operational \OGCM{}s in the \GODAE{} genealogy have intentionally focused on non-tidal ocean dynamics. This focus has been motivated by the separation of time and length scales that allows for the simulation of relatively slow turbulent features at the mesoscale as a computationally tractable problem that can be adequately constrained by along-track altimeter observations.
In this view, much tidal variability is just the fast propagation of long wave length noise; a background sloshing of the barotropic ocean state not obviously connected to the evolution of eddies and boundary currents.
Moreover, the skill of spatial harmonic tide models in representing surface tides over the deep ocean facilitates the use of de-tided or filtered altimeter observations to constrain a mesoscale non-tidal simulation. In effect, the fluid simulation is split across two qualitatively distinct modelling approaches; an LTI frequency-space tidal ocean and a time-stepped turbulent mesoscale.  
\BL{} is a ``non-tidal'' system in that \ATGP{} forcing is not applied, and in contrast to local downscaled simulations, there are no lateral open boundary conditions. Not only does the mesoscale simulation exclude the gravitational tidal body forcing, it also excludes the surface boundary application of atmospheric pressure.   
Surface pressure is excluded as a forcing term for analogous time and length scale reasons as the tides.  In fact, for the barotropic mode the pressure and tidal forcing can be formulated in an identical manner as adjustments to the spatial gradient of the surface elevation as in \citet[equation 9.9.5]{gill1982atmosphere}.

The state variable quantifying surface elevation carried by the model is thus a peculiar sea level anomaly (SLA) which in itself is a rather abstract quantity. 
Using the \citet{Stevens:2001kb} nomenclature, \BL{} simulates an aggregated pseudofluid system with a qualified relationship to the actual ocean.


From that background, recent publications indicate a motivation towards more dynamic representation of the effects of ocean tides.   This is indicative of pervasive goal across numerical simulation practice to increase model concreteness by reducing the role of parameterisations \citep[section 5.3]{Petersen:2012tr}; a reduction of system aggregation \citep{Stevens:2001kb}.\\
Similarly, Griffies describes the ``general trend in ocean climate modelling towards reducing many of the common approximations'' \citep[pp20] {Griffies:2004vs}.   It could be said that the modelling community perceives parameterisations as a compromise that should ideally be replaced with explicit physics when computing power allows.

% split 
The timescales targeted by a mesoscale simulation are slow relative to surface tides.
Given the focus on estimating the slower baroclinic evolution on the three dimensional ocean state, operational ocean models following \GODAE{} take numerical approaches that employ some special treatment for the faster barotropic hydrodynamics. 
%rigid ild
Earlier models avoided direct representation of surface height variations by making the so-called `rigid-lid' approximation \citep[pp128]{gill1982atmosphere} which leads to other ramifications for ocean forecasting discussed by \citep[pp19]{Griffies:2004vs}.
% split explicit
The ocean model underlying \BL{}, \MOM{} \citep{Griffies:2008vh}, does carry a free surface height as a state variable by employing a `split-explicit' timestepping scheme.    
In essence, this approach timesteps a computationally cheap shallow-water representation of barotropic dynamics many times for each relatively expensive update of the full depth-dependant ocean state.  
Thus, by representing the barotropic free surface dynamics of the ocean, a mesoscale \OGCM{} like \BL{} could in principle simulate the global tides.
But tidal forcing is not applied in \BL{} and there are several high level arguments against doing so.
   
%-----------------%
\subsection{Motivation to explicitly simulate tidal effects}

Excluding tides from a mesoscale ocean simulation is essentially an undesirable but necessary design choice found to facilitate operational forecasting.
Barotropic tides interact with the baroclinic structure of the ocean, which in turn modulates the behaviour of barotropic waves; but the question as to if and to what extent these interactions can be paramaterised or ignored must be answered within practical constraints.
But design constraints are re-evaluated periodically; especially in light of the ongoing improvements in computational capacity and observation coverage.
Computational costs have reduced dramatically over the last decade and there is a motivation  to more explicitly simulate the role of tides in \OGCM{}s.
This motivation is primarily directed at representing the influence of barotropic tidal motions on the baroclinic mesoscale structure of the ocean state; not at directly improving the representation of total sea level.

Overall the tendency of geophysical simulation is towards every higher fidelity and concreteness - as discussed above. But the particular targets for a which some explicit simulation of tides in \OGCM{}s are aimed include:
\begin{itemize}
    \item improved spatial distribution of vertical mixing;
    \item improved simulation of barotropic/baroclinic energy conversion;
    \item improved resolution of frontal features;
\end{itemize}
These aims do not necessarily motivate an accurate simulation of the sea surface elevation.
Rather, in the spirit of Figure \ref{fig:ogcmScales}, it is a configuration with regard to what processes are relegated to sub-gridscale parameterisations.

Thus in the \OFAM{} simulations of \citep{Schiller:2004fv},  it was reasonable to include only a small subset of tidal constituents and make tailered adjustments to better match the astronomical phase information.  In contrast to a standard tide prediction,  the aim in this case was ultimately to improve the broad spatial distribution in which tidal currents influence mixing and water mass formation.

\citep{Arbic:2004wz} has highlighted the `inordinately large' magnitudes allocated to drag parameterisations in tide simulations generally, and the fact that large scale tidal simulations have to somehow account for the important role of barotropic to baroclinic energy conversion.   This conversion must be fully parameterised for a depth integrated simulation, but for a tidal \OGCM{} with baroclinic dynamics the line between what is explicitly modelled and what is parameterised is redefined.  

Inclusion of tidal motions within the HYCOM \OGCM{} has advanced substantially and been shown to benefit the representation of certain baroclinic ocean structures  \citep{10.1016/j.ocemod.2019.02.008}.    But this trajectory towards higher fidelity still relies on careful and tidally-specific parameterisation methods to account for unresolved processes; such as by the depth-dependant M2 ``wave drag'' term \citep{Jayne:2001tr}.

Some further comments on the explicit resolution of tides in \OGCM{}s is included in Appendix \ref{appendix:globalTides}



To date, the inclusion of explicit tides in \BL{} remains far from the Australian operational schedule.
But given the trajectory of operational forecasting more widely, the question of how to handle the boundary between tidal and non-tidal simulations will require continual revision.
One operational reference point for handling questions about differing representation of tides across numerical systems is provided from outside the \OGCM{} context; with the design of the United Kingdom Storm Tide Warning Service \footnote{now titled UK Coastal Monitoring and Forecasting}\citep{Horsburgh:2008gw}.   In this design the weather forecast simulation includes tides in order to account for non-linear interactions between the tidal and nontidal sea level effects; however this total water forecast is not applied directly, but is differenced with a tide-only simulation to derive a forecast surge (or residual) signal.   This configuration allows for both the simulation of interaction effects but the continued use of an established or superior representation of the tidal component.


Arguably the challenge presented by higher fidelity operational ocean forecasts is in how to usefully exploit these inherently incomplete and approximate representations;  
\textit{while a model can never be “truth,” a model might be ranked from very useful, to useful, to somewhat useful to, finally, essentially useless} \citep{Burnham:2002}.
The ability to rank usefulness and design forecast products will always be a function of the application, regardless of progress towards ever more concrete simulations.

%-----------------%
\subsection{Representation of coastal boundaries}
The numerical representation of the land/sea boundary is of special interest for the present focus on coastal sea level.
Model design choices that allow for the tractable representation of mesoscale ocean circulation properties stand in contrast to those that would specifically target coastal processes.
The image in Figure \ref{fig:oceanmapsMaskDetail} illustrates how the \BL{} spatial grid discretizes the Australian coast into approximately 10km steps.  A more national view is shown later in Figure \ref{fig:map_masks}.
At face value this resolution may appear too coarse to usefully simulate the water level at a coastal location; let alone within an shallow embayment such as Port Phillip Bay shown in the image.    But the length scales that can be represented rather lead to an expectation that the useful timescales of the simulation will be limited in some related way.   Chapter \ref{chp:aggregate} demonstrates that there is qualified value at the scales represented.
%---------------------------
\begin{figure}[H]
    \includegraphics[width=\figwidthBig]{figures/images/oceanmapsMaskVic.png}
    \caption{Horizontal land/sea mask detail.}
            {Focus on region of south eastern Australia near Melbourne. The $0.1^{\circ}$ regular grid approximates the coast in ~10km steps and totally excludes features like Port Phillip Bay.}
    \label{fig:oceanmapsMaskDetail}
\end{figure}
%---------------------------
But for the present document there are details beyond a nominal length scale that are worthy of explication.

Firstly, \BL{} simulates the ocean state on a regular Arakawa-B staggered horizontal grid.
This grid carries a fixed allocation of the land/sea mask, so there is no time evolution of where the coast is located as there could be in a system with the feature commonly called wetting-drying.
Furthermore, the system uses a fixed bathymetry and takes no account of sediment transport or any other active change to the lower boundary.

The staggered B-grid complicates how the coastal boundary is represented as there are effectively two horizontal grids that each determine what is wet and what is dry.
In particular, the sea level anomaly quantity ($\eta_t$) is quantified on the `T' or tracer cell grid.    However, in order to integrate the shallow water equations forward in time information is exchanged with velocity values carried on the `U' grid; and furthermore each timestep involves a spatial operation over a stencil of neighbouring cell points. 
The numerical implementation in \BL{} of the barotropic hydrodynamics leads the existence of a so-call checkerboard null-mode, a numerical artefact that can allow aphysical spatial discontinuities in sea level to grow.   To mitigate this null-mode a spatial smoothing step is included within the fast barotropic integration. Figure \ref{fig:gridStencil} summarises schematically the grid stencil operators involved in the barotropic update cycle.
%---------------------------
\begin{figure}[H]
    \begin{center}
    \includegraphics[width=\figwidthBig]{figures/diagrams/mom_BT_stencils_extra.png}
    \caption{Arakawa-B horizontal grid stencil illustrations.}
            {These stencils are relevant to the baroptropic mode updates of \BL{} and indicate the abstracted representation of the coastal boundary. The full forecast sea level anomaly includes both barotropic and baroclinic contributions}
    \end{center}
    \label{fig:gridStencil}
\end{figure}
%---------------------------
These spatial considerations simply suggest that \BL{} should be able to simulate phenomena operating at length scales of several tens of kilometers.
But there are hydrodynamic processes that are inherently coastal, for which the relevant length scales are likely not isotropic and are more naturally cast in coordinates related to the coast itself.    
This topic is taken up in Chapter \ref{chp:waveguide} with regard to the representation of coastally trapped phenomena.




