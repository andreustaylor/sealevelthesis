%--------------
\chapter{Research questions}

The maturation of mesoscale ocean prediction systems including \BL{} has bought qualitatively new types of sea level prognosis into the same context in which traditional tide tables have long been produced.
Whilst the canonical numerical approach to coastal sea level involves the pursuit of simulations with ever-higher spatial resolutions, this existing setting motivates an investigation into signal representation and the value of larger and longer scales of predictability.

\BoxBegin
\em{What is at the intersection of conventional tidal analysis and mesoscale ocean prediction, and what are the ramifications for operational sea level forecasts?}
\BoxEnd

This desktop study addresses the question using data and forecast systems of the Australian Bureau of Meteorology. New methods for combining, evaluating and delivering nominally “tidal” sea level information are proposed and analysed; with a focus on representation issues and system compatibility.


Despite the relatively low spatial resolution of the coast can \BL{} skilfully forecast coastal sea level and with what limitations? 

How is the surface height signature of coastally trapped waves represented in operational forecast models? 

Can nominally non-tidal sea level forecast systems simulate tidal patterns and why?


How can harmonic tidal prediction be approached when one end-use is aggregation with dynamic models?


%--------------
\chapter{Introduction}

%--------------
% review
%--------------
\section{Research questions}
This thesis addresses conceptual issues at the intersection of tide prediction and mesoscale oceanography, with practical implications for operational sea level forecasting practice. 
In contrast to the canonical numerical approach to coastal sea level that broadly pursues  ever-higher resolution simulations, this work turns to focus on the less-studied but relevant value offered by established operational systems; especially at the ocean mesoscale. Addressing the representation of sea level at the intersection of existing systems is important for building the foundations of future multi-model seamless forecasting services.


\emph{%
Is the operationalisation of high resolution coastal simulations a prerequisite for providing regular sea level forecasts?  Despite comparatively low coastal resolution, can \BL{} directly provide useful prognosis of non-tidal coastal sea level and with what limitations? }



\emph{Are point-by-point tide-gauge observations the only relevant basis on which to compare and assess heterogeneous sea level forecast simulations? 
Why is the role of coastally trapped waves prominent in the oceanographic literature but seemingly absent from operational forecasting?}



\emph{Has numerical simulation rendered conventional tide prediction redundant for operations?  How can the unique traditional function of tide prediction compliment nominally non-tidal ocean forecasts and future seamless prediction systems? 
}


%--------------------------------------------------
\section{Observable sea level and forecasts}

Sea level plays a unique role in physical oceanography and forecasting thanks at least in part to its very \emph{observability} \citep{Wilson:2010hy}. 
Variations in coastal \emph{still water level} \citep{Pugh:2014di} reflect a diverse range of hydrodynamic phenomena operating across time and length scales.   Furthermore the relative significance of the pheneomena vary greatly in space and time.
\citet{10.1007/s10712-019-09531-1} recently reviewed this diverse mix of physical phenomena present in the observations and note that ``\textit{it is mistake to think that coastal sea level variability always has a ‘local’ forcing (although it often does).}''. 
Figure \ref{fig:sealevelScales} provides a schematic summary of the physical time and length scales of broad classes of these phenomena that contribute to ocean water level variations, taking inspiration from similar diagrams in \citet{10.1007/s10712-019-09531-1} and \citet{Chelton:2001ws}.
%-----------------------------
\begin{figure}[H]\centering
  \includegraphics[width=\figwidthFull]{figures/diagrams/scales_time_length.pdf}
  \caption{Schematic indication of the relative time and length scales of a range of physical processes contributing to observable coastal sea level.}
  \label{fig:sealevelScales}
\end{figure}
%-----------------------------
The visual differentiation of tides from the other phenomena in Figure \ref{fig:sealevelScales} is intentional.  Tides have unique properties; but exactly what is designated as tidal in an operational context can be more ambiguous than first impressions imply. This topic is unpacked further in section \ref{sec:tide_intro}. Tides in the schematic are seen to span from very short to very long length-scales but at more restricted frequencies than other phenomena.    Furthermore, the long period tides are are shown separately in anticipation of a later discussion of how these species are potentially problematic in some forecasting applications.

How the varying balance of phenomena is expressed in observed sea level is illustrated by the spectra shown in Figure \ref{fig:obsSpectraEg} for two Australian locations.   Variability at nominally tidal frequencies is a striking feature at nearly all coastal locations;  ``the [tidal] constituent lines emerge from the noise background as trees from the grass'' \cite{godin:1972}.
Observed power at tidal frequencies does not however provide a direct attribution to tidal forcing and the extent to which this matters for forecasting applications is a conceptual theme addressed in chapter \ref{chp:tideFlavours}.
%-----------------------------
\begin{figure}[H]\centering
	\subfloat[Darwin in Northern Australia - large diurnal tide.]{\includegraphics[width=\figwidthFull]{figures/plots/plot_14072.png}} \\
	\subfloat[Esperence in Southern Australia - powerful synoptic signal.]{\includegraphics[width=\figwidthFull]{figures/plots/plot_109504.png}}
	\caption{Spectral estimates from two coastal tide gauges. Hourly data, black = observations, red = tidal residual, nominally tidal frequencies are shown with shading. An example of a `mixed spectra' \citep{Percival:1998tw}, in the sense of that prominent tidal spectral lines appear to be embedded in a background continuum of coloured variability.}
    \label{fig:obsSpectraEg}
\end{figure}
%-----------------------------

Quantified observations of sea level currently available for operational use fall into essentially two categories: (1) insitu and (2) remote.    Figure \ref{fig:sealevelObsCartoon} illustrates this difference schematically.

Insitu observations from coastal tide gauges are by far the most established and studied source.    The availability of tide gauge observations is intrinsically connected to the history of sea level study and the development of forecasting techniques; the longer view of which is covered in the history of \citeauthor{Cartwright:2000tt}.
Typical tide gauge instruments in fact measure the `air gap' between a sensor and the water surface from which a relative level is derived \citep{PCTMSL-sp9}. 
But the very nature of why and how tide gauges are installed raises a list of special characteristics relevant to operational forecasting. These characteristics are variously considered to add complication or insight from the perspective of academic  differec
Coastal locations are subject to both global and spatially localised hydrodynamic effects; in addition to influences like sediment movement, riverine phenomena and even coastal engineering works.   
The limited extent to which the range of scales present in the tide gauge observations can be adequately represented by any forecast system is a fundamental characterisation discussed across chapters \ref{chp:aggregate},\ref{chp:waveguide}and \ref{chp:tideFlavours}.    
Tide gauge observations can be mapped to certain sea level impacts and decisions, though not always directly; as for instance in \cite{Hague:2019ha}. 
Commonly these observations are the basis of vertical reference information and geodetic derivations that are ultimately connected to the interpretation of any quantitative sea level forecast \citep{Woppelmann:2006un}\citep{AVWS2021}.
More comprehensive observations of sea level than are possible from tide gauges is listed as a key requirement for future development by \citeauthor{10.3389/fmars.2019.00437} and video-based systems in particular show promise \citep{10.1175/jtech-d-18-0203.1} and \cite{2018agufmep52d..26h}. 
%-----------------------------
\begin{figure}[H]\centering
  \includegraphics[width=\figwidthBig]{figures/diagrams/sealevel_cartoon.pdf}
  \caption{Simplified illustration of the relative nature of sea level and existing observing platforms.}
  \label{fig:sealevelObsCartoon}
\end{figure}
%-----------------------------
Remote observations of sea level are also derived from what is effectively a very large air-gap .   The derivation from active radar return signals to a meaningful sea level value is non-trivial but very mature away from the coast \citep{Fu:2001ub}.  Operational mesoscale oceanography is heavily dependant on altimeter observations across the global ocean and is discussed further in section \ref{sec:operational_oceanography}.  
Operational application of remote sea level observations near the coast is complicated by both the very presence of land within the sensing `footprint' as well as more powerful but shorter time and length scales of sea level variability \citep{Vignudelli:2011wl}.   Whilst this is a  rapidly advancing field, coastal altimetry effectively remains pre-operational and needs to solve the handling of fundamental details such as vertical reference datums \citep{10.3389/fmars.2020.549467}.
The active nadir radar architecture of satellite sea level observations of the last two decades is expected to undergo a qualitative change with the planned wide-swath altimeter missions starting with SWOT \footnote{\url{https://earth.esa.int/web/eoportal/satellite-missions/s/swot}, \url{https://auswot.org/}}.   Wide swath observations of sea level remote observations promise to radically impact water level forecasting in the longer term but can not hope to become operationally relevant for some years to come. 

Operationally-relevant observations of sea level suffer from additional limitations that can be handled differently in more academic settings.   The most basic limitations are whether observations are accessible in near-real-time and the extent to which appropriate metadata is available and up to date.  
Quality control of observations is challenging task that is applied differently between an operational setting as for important long term historical archive projects \footnote{such as \url{www.gloss-sealevel.org}, \url{uhslc.soest.hawaii.edu} and \url{www.gesla.org}}.

The de-facto network of tide gauges from which sea level observations are provided to the Bureau of Meteorology are operated by a range of port authorities, state agencies and the Bureau itself. Inhomogeneity of tide gauge data quality is expected to be a permanent feature of operations; a status that is argued to be a core value of traditional tide prediction services in chapter \ref{chp:tideFlavours}. All evaluations made against tide gauges in this thesis are intentionally based on the Bureau's own operational archive.

%--------------------------------------------------
\subsection{Unique perspective of the operational setting}
\label{S:operational_setting}

The Australian Bureau of Meteorology is an \emph{operational} forecasting agency with characteristics that fundamentally frame the present study. The operational setting involves:
\begin{itemize}
    \item a finite suite of foundational forecasting systems;
    \item a diverse range of downstream applications;
    \item reliance on real-time and imperfect observations;
    \item regular and continuous schedule of forecast production;
    \item a preference for reliability over fidelity;
    \item a slow development and system change cycle.
\end{itemize}
This setting provides a perspective on geophysical fluid simulations that thus differs from broader academia as well as more single-purpose commercial forecasting.   As a result it could be said that this thesis does not launch neatly from a single academic `conversation' \citep{Booth:2009vy} but rather will draw on both peer-reviewed literature as well as internal Bureau of Meteorology inputs where relevant.
Unfortunately operational documentation often falls far behind aspirations.  The practive..... TBC   ...Tidal observation manuals exist eg \citep{IOC:2005tj}, \citep{Level:2011wu}, \citep{Parker:2007wq}.  However, analysis methods and design justifications are less formally documented - if at all.  The IOC understates this situation simply: `many organizations have developed their own method of tidal analysis'\citep{IOC:2005tj}\\

\begin{figure}[H]\centering
  \includegraphics[width=\figwidthBig]{figures/diagrams/scales.pdf}
  \caption{Forecasting suite schematic with overlapping scales}
  \label{fig:SCALES}
\end{figure}



TBC - name why this raises PROBLEMS that are worth study ....

%--------------------------------------------------
\subsection{Relevance of restricted scope}

Compare to event based or behind real time studies ...


In contrast to the :
focus on extreme events
direction towards ever-higher resolution simluations.


Cite some on both global and Australian scale...


Viewing the
This thesis intentionally addressess the topic of routine forecasts of coastal sea level 
...
not to detract from the inevitable pursuit of ``comprehensive ...coastal '' 



\begin{figure}[H]\centering
  \includegraphics[width=\figwidthBig]{figures/diagrams/scales_focus.pdf}
  \caption{Focus on mesoscale and tidal scales overlap}
\end{figure}

set benchmarks and address some representation issues at the overlap of conventional tides and large scale ....

%--------------------------------------------------
\subsection{Preparing for seamless forecasts}

BoM research strategy.


More general:

\begin{itemize}
  \item Towards 'concrete' , increasingly realistic and inclusive primitive equation forward models.
Peterson M1 ]\citep{Petersen:2012tr}
 \item Towards data-driven, service target - idealised, simple and robust 
\end{itemize}




Tidal harmonic methods have for many decades provided the only routine source of sea level forecast - and with great success.  
In contrast, time stepping primitive equation dynamic models and data assimilation are relatively new arrivals to operational centres.  
These represent two qualitatively different perspectives on sea level prediction.\\


Over the past decade, the operational evolution of \OGCM-based prediction has increasingly brought the two approaches into something akin to Gallison's `trading zone' \citep{Galison:1996uc}.
Developments promise only to increase the amount of overlap into the previously independent practices of harmonic analysis, eg \cite{Arbic:2010us}.\\
Here Munk and Cartwright's aphorism from the 1960's regarding tidal analysis is illustrative:
\begin{quote}
  \dots predicting and learning are in a sense orthogonal, and the most interesting effects are those that cause the most trouble with a forecasting: the continuum, the nongravitational tides, and the non-linear interactions.\citep{Munk:1966ts} 
\end{quote}
Forecasts from \BL{} and other similar systems now represent sea level attributable to exactly these troublesome areas; or at least some physical subset therein.
Jayne's reference to the ``vexing problems'' \citep[pp812]{Jayne:2001tr} arising from applying frequency-domain theory in time-domain numerical tide modelling is illustrative.

\begin{figure}[H]\centering
  \includegraphics[width=\figwidthBig]{figures/diagrams/spectra_cartoon_1.png}
  \caption{Conventional decomposition concept: mixed sea level spectra comprising discrete tidal `lines' and red turbulent continuum.  Implementation of this concept has ramifications for sea level forecasting.}
  \label{fig:SPECTRA_CARTOON}
\end{figure}

Viewing sea level as a stationary set of harmonics is at face value quite at odds with a view of the ocean as a turbulent fluid.   In that sense, Gallison's portrayal of physics as `neither unified nor splintered into isolated fragments' \citep[pp 782]{Galison:1987wh} is apt.  Thus also is Peterson's assertion of `\dots{} the fundamental plurality of \dots{} scientific practice' \cite{Petersen:2012tr}. 

\include{chapters/intro-operational}
\section{Ocean tide prediction}
\label{sec:tidesOverview}
%-----------------%
\begin{quote}
Tidal prediction is the oldest form of ocean prediction, and is still the most accurate \citep{Parker:2007wq}. 
\end{quote}
%-----------------%
\subsection{Why it is worth unpacking ocean tides}
\label{sec:semantics}
The topic of ocean tides is particularly rich in ambiguous and arcane terminology.  
Moreover, it is apparent that this terminology and the underlying concepts lead to miscommunication within operational forecasting.  And as operational centres like the Bureau of Meteorology increasingly bring these conventional tidal concepts into the same setting as geophysically `concrete' simulations (introduced in section \ref{sec:concrete}) the need for clarity is amplified.  This potential for confusion is motivates the following exposition of relevant concepts built into the practices of ocean tide prediction.
\newline{}


Tidal methods of analysing and forecasting the ocean have a long legacy in the history of western science.  Economically significant tidal sea level predictions in fact pre-date the whole modern scientific enterprise, and the evolution of the tidal perspective mirrors much of the story of classical continuum physics.

The \cite{Cartwright:2000tt} telling of the history of ocean tide science is illustrative. Remarkably, and perhaps typically, this scientific history of tides never even attempts to pin down a definition of what tide really should mean.  This surely deliberate exclusion stands in stark contrast to the minutiae of historical approaches and formulations and that the history covers.  The absence of definition may be read as an assertion that the \emph{history itself} is the only meaningful way to frame the scope of tidal science.
\cite{Pugh:1996uz} at least addresses the question of definition explicitly and in doing so casts a very wide net that catches almost anything that can be considered periodic.
\begin{quote}
Although any definition of tides will be somewhat arbitrary, it must emphasize this periodic and regular nature of the motion, whether that motion be of the sea surface level, currents, atmospheric pressure or earth movements. We define tides as periodic movements which are directly related in amplitude and phase to some periodic geophysical force \ldots.
\end{quote}
It is notable that whilst for Pugh \emph{periodicity} is a key feature of anything that is `tidal', he does not lock this periodicity to astronomy.  Pugh's pragmatic definition appears to be designed to best accommodate the conventional methods of harmonic analysis and sea level prediction that are so well established in operational agencies. Whilst this periodicity-stressing definition sensibly reflects the context of Pugh's text, it is certainly not the only working conceptualisation of what constitutes ocean tides; either within the operational forecasting context or in academic settings. 

In contrast to the periodic definition, most modern authors place the tidal potential in a defining role. Such an emphasis on the tidal potential or Astronomical Tide Generating Forces \ATGP{} reflects a greater regard for dynamics and inputs, as opposed to observed outputs, that is more complimentary to the wider field of geophysical fluid dynamics.  
\cite{Hendershott:1981ub} treats ocean tides as that subset of oceanographic long waves driven by the \ATGP{}.  Notably his discussion is careful with the distinction between dynamics and `practical tide prediction'. But even Henderscott's primarily dynamical perspective is imbued with the cultural interplay of physics and pragmatic prediction; as illustrated by his inclusion of the rather aphysical radiational potential concept associated with the \cite{Munk:1966ts} response method tradition.
The Australian Tide Manual \cite{PCTMSL-sp9} similarly reflects the historical intertwining of tidal physics and practical prediction methods; such that any forcing physics simply provides a backdrop to the singular focus on harmonic prediction.   
A step removed from ocean prediction, \cite{10.1016/b978-0-444-53802-4.00058-0} provides a modern physical perspective that both emphasises the core role of the tidal forcing (in this case for earth tides) whilst adopting the well established terminology derived from the long history of ocean tide prediction. A similar emphasis on the physical forcing within the historical perspective is taken by \citep{Flinchem:2000kp} in a  discussion of analysis methods associated with non-stationarity in ocean tide patterns.
The centrality of \ATGP{} be the approach employed in the present work as well.


Further on semantic issues, it is worth highlighting the extent to which the subject of ocean tides raises ostensibly oxymoronic terminology.  
Geophysists can define an \emph{a-periodic} pole tide on the basis of the dynamic connection to gravitational body forces associated with astronomical motions.   And valid discussion of \emph{non-stationary} tides \cite{Ray:2011tj}, \emph{quasi-periodic} tidal phenomena \citep{Flinchem:2000kp} or \emph{storm} tides \cite{Horsburgh:2008gw} in the literature and operational products each conceptually clash with the perspective of conventional harmonic analysis.  
A comparable semantic disconnect can arise in circumstances where a tidal prediction exceeds the designated Highest Astronomical Tide.   
Each of these definitions are sensible in context, but can potentially be a source of miscommunication with the development and delivery of operational forecasts.    Chapter \ref{chp:tideFlavours} proposes measures to partially mitigate such problems.

%-----------------%
\subsection{Common foundation of the \ATGP{}}
Reference to the astronomical tide generating potential (\ATGP{}) is here considered to be the common foundation what is properly considered a tidal method or tidal phenomena.  
The \ATGP{} is an abstraction developed from positional astronomy and classical gravity theory alone.   Perhaps surprisingly for some readers, the underlying role of positional astronomy is formulated in an entirely geocentric, effectively pre-Copernican, framework. 
Figure \ref{fig:tideForceFlow} schematically illustrates the conceptual connections from positional astronomy, to the \ATGP{} and geophysics through to observed sea level .  
Differing treatment of the intermediate parts of this flow chart, the geophysical modelling, effectively comprise the alternative approaches to sea level forecasting.  Of special relevance to the present focus are the details of any decomposition of observed sea level into tidal and non-tidal categories.   The dotted lines connecting nominally non-tidal components indicate that the gravitationally forced response is not the sole factor in discussions of tidal sea level. 
%-----------------
\begin{figure}[h]
    \begin{center}
    \includegraphics[width=\figwidthFull]{figures/diagrams/tidal_force_flowchart.pdf}
    \caption{Ocean tide flow chart (following Agnew \citep{Agnew:2011ub}.  Reference to the \ATGP{} is common to tidal analysis and prediction methods, whilst the treatment of tidal/non-tidal connections can differ markedly.}
    \label{fig:tideForceFlow}
    \end{center}
\end{figure}
%-----------------
Before expanding on the role of the tidal forcing, the recent work of \cite{10.1016/j.oceaneng.2020.107013} is worth mentioning.   Regardless of the finer details, this observation-based machine learning application to tide prediction is still founded on the connection between positional astronomy and sea level, but simply makes the connection more indirect by relying on easily accessible moon phase data.
%-----------------i
\subsection{Basic development of the \ATGP{}}  \label{sec:basic_potential}
The centrality of the tidal generating potential to sea level forecasting warrants further elaboration, in order to later explicate the manner in which conventional and dynamic sea level forecasts respectively represent tides. 

The \ATGP{} is a mathematical abstraction founded on a consideration of the classical gravitational field near the Earth surface. The temporal variations of gravity in the vicinity of this surface are developed as a function of the \emph{geocentric} relative positions of the celestial bodies.
For ocean tide applications, only the two celestial bodies, the moon and sun, are considered relevant on the basis of relative contribution to the perceived gravity field changes. It follows that the information required for computation of this lunisolar tidal potential is encapsulated in the celestial positions (ie ephemeris) of the moon and sun alone \citep{Agnew:2011ub}.


The full gravity field is defined as a scalar potential in space fulfilling the Laplace equation $\Delta V=0$ \citep[sec 5.3.1]{Urban:2013vl}.  The spatial field $V(\theta,\lambda,r)$ can be formulated in spherical geocentric (ie fixed earth) coordinates as a weighted sum of surface spherical harmonics. As a potential field, contributions from each mass element can be computed separately and linearly superposed.

The specific subset of $V$ attributed to the celestial bodies external to the Earth, but excluding components acting uniformly over the Earth's surface, is defined to be the tidal potential \ATGP{} or $V_T$.
The associated tidal acceleration at any particular point on the earth surface can also be thought as the vector difference between the direct attraction of each celestial body and the orbital acceleration about the Earth-Body barycenter \citep{Wenzel:1997kn}.


The subset $V_T$ of $V$ that is relevant to ocean dynamics is formulated in geocentric coordinates following the convention of \cite{Cartwright:1973em} hereafter \CTE{}, and more recent notation of \cite{Desai:2006wo}:
\begin{equation}
    \eta_{eq} = \frac{V_T(t,\theta,\lambda) }{g} = \sum_{n=2}^{\infty} \sum_{m=0}^{m=n} M_{nm} P_{nm}( \sin(\theta) ) \text{Re} \left [ c^{*}_{nm}(t) e^{im\lambda} \right ]
    \label{eq:VT}
\end{equation}
Where the potential is described only on an idealised spherical earth surface in terms of time, longitude $\lambda$ and latitude $\theta$, as a sum of functions described further below. 

$P_{nm}$ are the associated Legendre polynomials of degree $n$ and order $m$.  Writing $P_{nm}(\sin(\theta))$ gives the surface spherical harmonics.  The fact that the sum begins at $n=2$ is discussed below.
$M_{nm}$ are normalisation factors, that whilst not of direct interest here are noted to follow different conventions between applications \citep{IERS2003}.
Figure \ref{fig:VTmaps} provides a visualisation of the field and a decomposition into spherical harmonics a single snapshot in time.
%-------------------------------
\begin{figure}[h]
    \begin{center}
    \includegraphics[width=\figwidthBig]{figures/maps/tidal_potential_spatial_20130101_0000.png}
    \caption{Snapshot of global $\eta_{eq}$ field illustrating spatial decomposition into spherical harmonics.  Note the wide variation of relative magnitude. When comparing to formulations recall that those decompositions are per body; ie an independent set for the moon and sun. }
    \label{fig:VTmaps}
    \end{center}
\end{figure}
%-------------------------------
Equation \ref{eq:VT} and the visualisation in \ref{fig:VTmaps} quantify the potential $V_T$ normalised by the standard value for Earth gravity $g$; a quantity defined as the \emph{equilibrium tide} $\eta_{eq}$.  
A convenience of normalising the potential field as $\eta_{eq}$ is that the quantity has units of height. But this formulation unfortunately invites confusion with actual ocean elevations; whereas the equilibrium tide really only has a very abstract and indirect connection to observed ocean tide heights.
Specifically, whilst useful for formulating the driving force, the any direct connection to ocean response is only via the thought experiment of a shallow water idealisation of boundary-free ocean with ``no dynamic effects'' as stated by \cite[Eq 9.8.3]{gill1982atmosphere}; that is, the equilibrium tide is a useless approximation to ocean heights at all expect the longest periods \cite{Egbert:2003jd}.


Equation \ref{eq:VT} represents the spatial decomposition of a surface potential field over the globe.  Of itself, such a formulation asserts nothing specific about the causation of the field.  It embodies no astronomy or ocean dynamics.
Practical implementation of \ref{eq:VT} requires choices regarding the set of harmonics $(n,m)$ to be included and the determination of the time varying complex weights $c_{nm}(t)$.


Only a small set of $(n,m)$ is taken to be relevant for ocean tidal dynamics; in contrast to the thousands of terms utilised for some earth gravity studies. What counts as a relevant ocean subset is generally determined on the basis of the relative magnitude of the terms and the nature of the temporal variation in geographic coordinates.

It is only the horizontal gradient of the potential $\nabla V_T$ that can drive changes in the distribution of ocean mass.   Which to be clear states that the effect of the \ATGP{} on the ocean is to slide mass side-to-side, not directly pull it up.   And furthermore, only \emph{temporal} changes in this horizontal gradient will effect the non-static ocean distribution. Subsequently degrees $n=0,1$ are not relevant to ocean tides.   This is represented in equation \ref{eq:VT} by the lower limit $n=2$ of the outer sum.
For almost ocean tide applications an upper limit of $n=2$ is taken to be sufficient, or at most $n=3$; again in contrast to some earth gravity studies.  This choice is based on the rapid decline in relative magnitude of each harmonic with increasing $n$ for the luni-solar system - discussed below regarding equation \ref{eq:c}.



The zonal harmonics $(n,m) = (2,0)$ for each body are of particular relevance to the slower patterns of sea level.  Specifically, by having no variation in $\lambda$ these harmonics don't vary geographically with the daily rotation of the earth.   However, the slower changes in the declination of each body do drive an ocean mass response and is associated with the both the long period and permanent tides.  
What is considered to comprise the \emph{permanent} component of the tide potential is effectively time scale and application dependant and is an important detail for geodesy and gravity studies.  \cite[section 5.3.3.2]{Urban:2013vl}.  For ocean forecasting, the role and formulation of the permanent tide becomes relevant in the decomposition of mean sea level, and estimation of what constitutes dynamic ocean topography in a consistent geodetic reference framework \citep{Filmer:2018cu}\cite{10.1007/bf02520477}.


All of the temporal variation of $V_T$ is contained in the time series of complex scalar weights $c_{nm}(t)$.
It is significant that these temporal variations of the \ATGP{} for the entire globe can thus represented by a small number of scalar timeseries - a single complex timeseries for each spherical harmonic included.  For the typical set of harmonics used for ocean tides $(n,m)=(2,1),(2,2)$ this represents a significant compression of spatial information.  
Given the positions in the geocentric reference frame $\theta,\lambda,r$ for both moon and sun, the coefficients are:
\begin{align}
\label{eq:c}
c_{nm}(t) &= a_{nm}(t) + ib_{nm}(t) \nonumber \\
          &= \sum_{b=\text{moon},\text{sun}}    \frac{4 \pi GM_{b}}{g r_{b}}  \frac{(2-\delta_{m0})} {(2n+1)} \left(\frac{a}{r_b} \right)^n    M_{nm} P_{nm}( \sin(\theta_b) ) e^{im\lambda_b}
\end{align}
The normalisation used in the determination of $c_{nm}(t)$ must be consistent with that used for equation \ref{q:VT}.
Note that $\theta,\lambda$ in Equation \ref{eq:c} are equivalent to the geographic coordinates of the respective sub-body point at a given time. 
Where $\delta_m0 = 1$ for $m==0$ and $\delta_{m0} = 0$ for $m \neq 0$.\\
Normalisation $M$ is given by Equation \ref{E:M}. The radial scale $a$ is conventional taken as the semi-major axis from the ellipsoidal georeference. 
%---------------------
\begin{figure}[h]
\begin{center}
\includegraphics[width=\figwidthBig]{figures/plots/tidal_coeff_timeseries_2days.png}
\caption{Snapshot of time varying coefficients $c_{nm}(t)$.  In the upper panel, classification of $c_{2m}$ into long (red) , diurnal (blue) and semi-diurnal (black) species is evident.  Note the much smaller magnitude of higher degree harmonics.}
\end{center}
\end{figure}
%---------------------
The rapid decay of the scaling term $\left(\frac{a}{r_b} \right)^n$ with n is the basis for excluding higher degree harmonics from ocean tide applications.  In the case of the Moon, the magnitude of the $n=3$ potential field is more than 2 orders of magnitude smaller than $n=2$.  The associated force also decays, but not quite so rapidly due to the decreasing spatial length scales of the higher harmonics.  This decay is apparent in the colorbars of Figure \ref{fig:VTmaps}\\
The same relative magnitude argument is applied to neglecting more distant celestial masses from formulation of $\eta_{eq}$.

%-----------------%
\subsection{Primary role for temporal variations}
\label{sec:temporal}
The temporal variation of the \ATGP{} is of special relevance to ocean tide prediction.  And inspection of equation \ref{eq:VT} show that all of this variability is contained within the coefficients $c_{nm}(t)$.  
Any tidal method that applies $c_{nm}(t)$ from a numerical ephemeris in time-space could be said to be \emph{direct}. 
However, transformation of $c_{nm}(t)$ into frequency-space is in fact the basis of many tidal methods.   The highly clustered frequency content of $c_{nm}(t)$ renders this transformation particularly useful.  
Transformation of $c_{nm}(t)$ into frequency space is the basis for \underline{harmonic developments} of the \ATGP{}. Whilst there have been different approaches to performing this development, the common representation is given in Equation \ref{E:harmonic} following \citep{Desai:2006wo} and \citep[Eq 13]{Cartwright:1971iz}.
Furthermore, the ephemeris details for the celestial bodies can be modelled with relatively simple polynomial functions of time; discussed further below. 


It is this harmonic decomposition of the \ATGP{} that leads to the conventional ocean tide practice of harmonic analysis.
%----------------
\begin{equation}
    c_{nm}(t) = \sum_{k} H_{nmk} e^{-i( t\theta_{nmk})} = \sum_{k} H_{nmk} e^{-i( t\omega_{nmk} + \beta_{nmk})}
    \label{E:harmonic}
\end{equation}
%----------------
The index $k$ represents a discrete series of tidal components; potentially expanded out to hundreds of terms.  Each component is associated with a discrete point in complex frequency-space with amplitude $H$, frequency $\omega$ and phase $\beta$.

Despite a common misconception, it is important to emphasise that equation \ref{E:harmonic} does not represent a Fourier series.  The finite sequence of frequencies $\omega_{nmk}$ are derived from the relative motions of the earth-moon-sun system and do not provide either an orthogonal set of sinusoids nor a complete basis.

Despite not being orthogonal, the harmonic decomposition does provide a special list of frequencies that can be ranked by a scalar magnitude. 

\citet{Doodson:1921kt} introduced a novel and influential system of notation for specifying $\theta_{nmk}$ based on a laborious harmonic decomposition of $V_T$ using Brown's lunar theory; that is, a polynomial model for the lunar ephemeris.  
In the Doodson formulation, all the relevant astronomical information is summarised by code of 6 small integers; together called the Doodson number or argument numbers.   Each position in the code is associated with an fundamental astronomical concept as summarised in Table \ref{T:doodson}.  Doodson codes are in common use across the tidal literature and provide the only useful means of describing tidal components beyond the small number associated with traditional Darwin names such as M2, K1, O1, S2.
Of these traditional names \citet{10.1016/b978-0-444-53802-4.00058-0} correctly states that `` [whilst] it is convenient to have a shorthand way of referring to these harmonics; unfortunately, the standard names,  now totally entrenched, ... simply have to be learned as is.''.

There has been some variation of conventions with regard to the exact formulation of Doodson numbers.  One such detail is the avoidance of negative integers in the code by the addition of the arbitrary constant 5 to all integers except $d_1$.   A less common variation involves the use of solar-hour in place of lunar hour $\tau$.  
In essence the Doodson codes provide a compact unique specification, or in practice \emph{definition} of frequencies relevant to tidal methods.

Whereas the Doodson codes provide a reliable way to describe these tidal frequencies, the mapping to traditional names is unfortunately not always consistent and can lead to errors if transferring parameter data between software platforms; this issue is taken up in chapter \ref{chp:tideFlavours}.


\begin{table}[htp]
\caption{Doodson astronomical arguments.  Small integer combinations of these six numbers $d_1 d_2 d_3 d_4 d_5 d_6$ are used to classify tidal components.  Recall that longitudes are celestial, not geographic, coordinates}
\begin{center} 
\begin{tabular}{|c|c|c|}
\hline
Description                            & Argument          & Period\\
\hline
Mean lunar hour                        & $\tau$            & $\sim$ 1 day      \\
Moon mean longitude                    & $s$               & $\sim$ 27 days    \\  
Sun mean longitude                     & $h$               & $\sim$ 1 year     \\
Longitude of lunar perigee             & $p$               & $\sim$ 8.85 years \\
Negative longitude of mean lunar node  & $N^\prime$        & $\sim$ 18.6 years \\
Longitude of Sun mean perigee          & $p_1$             & $\sim$ 20000 years\\
\hline
\end{tabular} 
\end{center}
\label{T:doodson}
\end{table}


Equation \ref{E:doodson} gives the angular argument for a single tidal component $k$ as a function of the 6 astronomical arguments.   As discussed above, the astronomical arguments embody an analytical ephemeris for the Moon and Sun rather than a direct numerical description of these positions.  Polynomial functions of time can be used to estimate each of the 6 arguments close to a given epoch.  The phase adjustment $\delta$ is a convention applied such that the each term in \ref{E:harmonic} is written as a cosine, rather than the mixture of sine and cosine terms that naturally follow from the underlying spherical harmonics.

\begin{align}
\label{E:doodson}
\theta_{nmk}  &= \left[ d_1 , d_2 , d_3 , d_4 , d_5 ,d_6 , \delta(n,m)  \right] \cdot \left[ \tau , s  , h , p , N^\prime , p_1 , \frac{\pi}{2}   \right]   \\
          d_1 &\equiv m \nonumber \\
              & \mbox{ (nb ignoring integer offset 5 often added to $d_2d_3d_4d_5d_6$)} \nonumber
\end{align}

\begin{equation}
\delta(n,m)  =     \begin{array}{ll}
                    1 & \mbox{if $n+m$ odd}  \\
                    0 & \mbox{if $n+m$ even} 
                    \end{array}             
\end{equation}


%-----------------%
\subsection{Other aspects relevant to global coordinates}
\label{S:ATGP_extras}
The essential development of the \ATGP{} in Section\ref{sec:basic_potential} is standard.  However, progressing beyond the basic development towards an ocean forecasting application raises several issues of significance that depend on the details of the application.


\underline{Solid Earth deformation and vertical reference.}  \\
The basic spatial perspective behind the development of the \ATGP{} is the thin shell approximation.  Geocentric coordinates are used.  This is appropriate for writing the potential in general form.  However, when considering the response of ocean dynamics, a vertical relative to the ocean floor is dynamically relevant.  There is a vertical movement of the ocean floor in geocentric coordinates associated with the \ATGP{} of comparable magnitude to the ocean tide which is quantitatively significant to dynamic models \citep{Hendershott:1981ub} and \citep[pp.336]{gill1982atmosphere}.


This elastic response of the earth can considered to be essentially instantaneous compared to ocean timescales.  Furthermore, the elastic re-distibution of mass associated with this 'earth tide' itself modifies the gravitational field.    By treating the Earth as "spherical, non-rotating, elastic, and isotropic" the solid body response can be encapsulated by a small number of dimensionless Love numbers \citep{Agnew:2011ub}.  In the present context each spherical harmonic degree has a single body tide Love number $h_n$ and `induced free space potential' Love number $k_n$ \citep[Sec 5.3.3]{Urban:2013vl}. This has the convenience of formulating the combined \emph{direct} effects of the solid earth response to the \ATGP{} as a modification to the magnitude of $\eta_{eq}$.  It is noted that ``the Love numbers for the diurnal tides differ from those for the semi-diurnal and long-period tides because of the free-core nutation resonance'' \citep{Arbic:2004wz}.
The conceptually related, but more complicated, effects of the moving ocean mass itself is discussed below.

\begin{equation}
\eta_{eq} = -(1+k_n-h_n) \frac{V_T}{g} \sim 0.7 \frac{V_T}{g}
\end{equation} 


\underline{Treatment of self attraction and loading.} \\
The tidal movement of ocean mass has an effect on the solid Earth, and reflexively on the gravitational potential acting on the ocean itself.
From an ocean modelling perspective, elastic compression of the solid earth due to the time varying mass of ocean is referred to as the `load tide'.  The effect of the moving distribution of ocean mass on the gravitation potential field is referred to as `self-attraction'.

Together these effects are lumped together as self-attraction and loading \SAL{}.  They are conceptually distinct from body tides in that \SAL{} is a reflexive function of the time varying spatial distribution of global ocean mass.
Similar effects that vary on tidal timescales are also associated with the time varying distribution of atmospheric mass as well land-based loading from ice,snow and soil moisture.  The value of explicitly distinguishing non-ocean \SAL{} in the context of ocean forecasting is not known, and no publications appear to address this in detail.  It is likely that practical evaluation of \SAL{} for ocean modelling implicitly accounts for some atmospheric effects - especially for the ocean tide associated with pressure forcing S1 and S2.


Similarly to body tide effects, it is possible to formulate \SAL{} as a scaled modification to the \ATGF{}.  This is very convenient, but known to be associated with significant inaccuracies \citep{Ray:1998jl}.  The present distribution of \MOM{} makes this so called `$\beta$' approximation.\\
An evaluation of intermediate methods to parameterise SAL is described by Stepanov and Hughes \cite{Stepanov:2004up}
In contrast to the $\beta$ scaling approximation, it appears to be quite reasonable to consider \SAL{} as separate pre-calculated body forcing rather than a scaling of $\eta_{eq}$. 
\begin{quotation}
SAL should be computed by convolution \dots with the Green's function for loading and self-attraction. Since this convolution smoothes out small-scale features, and since large-scale tidal elevations are now well determined over most	of the earth, SAL is in fact now reasonably well known, even where local details of tidal elevations and currents remain uncertain. \citep{Egbert:2002ug}
\end{quotation}


\underline{Time, polar motion and ephemeris.}\\ 
Diurnal Earth rotation and hence time UT1 are effected by tidal effects \citep[sec 8]{Anonymous:2004tm}.  In addition, conventional UT1 contains discontinuities at leap seconds is not strictly identical to ephermeris time.  For the purposes of ocean forecasting these variations are in general considered negligible and definitions of time are treated as unproblematic.\\


A special case is the \underline{pole tide}. Geocentric coordinates align the polar zenith approximately with the Earths axis of rotation. However, there are continual variations of alignment from the reference axis described by theories of precession and nutation and in general referred to as `polar motion'.   More generally, any changes in the Earth's instantaneous rotation vector can be associated with changes in the gravity field  experienced in surface fixed geocentric coordinates.\\
\begin{quotation}
Polar motion of the Earth is almost completely described by two harmonic variations of the location of the instantaneous rotation pole with respect to the mean rotation pole: an elliptical motion at an annual period, and an almost circular motion at a period of 14 months. The 14-month variation is a free mode of the Earth referred to as the Chandler wobble and has amplitudes that vary with time. \citep{Desai:2002ev}
\end{quotation}
The gravitational effect due to polar motion can be formulated similarly to the \ATGF{} as a potential field decomposed into spherical harmonics.
Pole tide effects are included as corrections to altimetry products, but are ignored in standard tide table production.\\
Another phenomena effecting Earth rotation is the Free Core Nutation (FCN).  For the present ocean tidal purposes the FCN will be considered relevant only with regard to the solid earth tides.  The FCN is a factor in defining the  Love numbers in the diurnal band.   Related to this, Desai and Wahr note that the definition of the K1 $[1 1 0 0 0]$ input amplitude used for solid earth tide algorithms can be inappropriate for ocean applications if compensation for the effects of the FCN resonance are included\cite{Desai:1995je}.



The formulation of the tidal potential in Section\ref{sec:basic_potential} involved only the instantaneous relative positions of the Earth-Moon-Sun system.   These positions are taken from an ephemeris.  In practice, there are two broad classes of ephemeris; analytical and numerical \citep{Wenzel:1997kn}.


An analytical ephemeris provides the relative positions of the Moon and Sun via relatively simple polynomial relations, optimised for a given epoch.  Such analytic ephemeris have been commonly employed for ocean tide studies.   Analytic ephemeris were used initially for computational necessity and more recently for computational convenience.


For modern \emph{astronomical} purposes numerical ephemerides are now standard.  Prior to 1984 standard ephemeris were based upon \emph{theories}\cite[sec 8.1]{Urban:2013vl}, whereas contemporary ephemeris are created by the application of dynamical equations integrated into the future, after initialisation via assimilation of past observational data.

At the time of writing, the current best estimates for the orbits of the Moon and planets is DE421 \citep{Folkner:2008wm}.   The snapshot visualisation shown in Figure \ref{fig:VTmaps} were calculated based on the application \ref{eq:c} to output of DE421.


The direct application of modern numerical ephemeris to global ocean tide simulations has been demonstrated by for instance \cite{Weis:2008ex} and \cite{10.1007/s10236-016-1016-1}.
However the attractive properties of this approach appear yet to have outweighed the benefits of conceptual and implementation simplicity offered by conventional decomposition approaches.   The ability to isolate named constituents for evaluation is valued for inter-comparisons such as that of \cite{Stammer:2014vh}
A more accurate ephemeris is likely of less significance to ocean tide hydrodynamics than the conceptual approach regarding the extent to which forcing and response are viewed in frequency space \citep{10.17125/gov2018.ch13}.

Considering the ocean response to tidal forcing in frequency-space is arguably just as foundational to tide prediction and sea level studies as the \ATGP{}; and this topic is addressed next.  

%-----------------%
\subsection{Ocean as an LTI System}
\label{sec:LTI}
With historical hindsight it could be said that the success of tidal sea level prediction have been based on treating the ocean as a linear time invariant (LTI) system driven by the \ATGP{}.\footnote{The nomenclature LTI is taken from engineering control theory , and not the historically influential Liverpool Tide Institute.}

In essence, the ocean response to tidal forcing is modelled as stationary in frequency space.  Once this time-invariant ``black box'' model has been characterised, astronomical empherides map linearly to sea level predictions.  The details of how the LTI system is characterised and implemented distinguish the variants of tidal analysis and prediction.

Two important aspects of this broad approach were introduced in the late 1700's by Laplace \cite[chpt 7]{Cartwright:2000tt}:
\begin{itemize}
    \item the spectral banded-ness of $V_T(t)$ and the value of a stationary frequency perspective;
    \item application of semi-empirical analysis to simply characterise the expression of complex hydrodynamics. 
\end{itemize}

Modelling the ocean as an LTI system in this manner necessarily takes a frequency-space perspective and assumes model stationarity. \citet{Jay:2003bj} suggest that that the long history and apparent value of this perspective has $\dots$
\begin{quotation}
solidified an opinion that tidal time series (particularly those of surface elevation) are basically stationary, with non-stationary components frequently being regarded as meaningless `noise'.    
\end{quotation}
In fact, as discussed in section \ref{sec:semantics}, frequency-space stationarity is essential to the \emph{definition} of what is considered tidal in most operational settings.
% little flow chart
\begin{align}
    c_{nm}(t) \Rightarrow & \fbox{Global Ocean}\Rightarrow \mbox{observed response} \nonumber
\end{align}

Time invariance of the LTI in frequency-space means that each input component maps directly to an output response at the same frequency, characterised by an amplitude and phase transformation.   As the LTI characteristics are identified via semi-empirical methods, any hydrodynamics are essentially hidden within the black box and are largely irrelevant.

The core of such a tidal model is schematically shown as a mapping at each distinct tidal frequency:
\begin{align}
    H_{k}(\theta_k) \Rightarrow & \fbox{empirical LTI system} \Rightarrow \mbox{tidal ocean $f(\theta_k)$}  %\nonumber
    \label{eq:LTI}
\end{align}

Conceptualising the global ocean in frequency space raises some peculiarities as characterised by  \citet{Groves:1975ky}: ``by an unfortunate coincidence, the size of the earth, the depth of the ocean, and the rate of earth rotation are such that the frequencies of the free modes of ocean oscillation are intercalated with the frequencies of the tide-generating consitituents''.
For sea level, the different forecasting methods are essentially variants on the  definition of a reference input signal and manner in which the LTI system is formulated.   Following Munk and Cartwright \citep{Munk:1966ts} there are three foci for the problems associated with this conceptualisation:
%\begin{inparaenum}[(i)]
\begin{itemize}
\item the spectral continuum of ocean variation;
\item non-gravitational phenomena; 
\item non-linear interactions.   
\end{itemize}
%\end{inparaenum}

%-----------------%
\subsection{Tidal Harmonics and Formalisms}
\label{sec:formalisms}
There is no unique approach to implementing the LTI concept of ocean tide prediction. In fact, ``\textit{many organizations have developed their own methods of tidal analysis}''\citep{IOC:2005tj}. But the different approaches or formalisms share the LTI system model discussed above.  
In very broad terms tidal these formalisms fall into basically two categories: harmonic methods and response methods.  
Neither of these concepts actually involve any hydrodynamics. When hydrodynamic simulations of the ocean are undertaken it seems that tidal LTI conceptualisations are at least indirectly involved in one way or another; and this will be apparent in subsequent sections dealing with hydrodynamic and spatial models. 

Within operational authorities, \emph{tide prediction} is practically synonymous with conventional harmonic methods. And it is here that the role of special frequencies (section \ref{sec:temporal}) has come to play a primary role.
``Standard harmonic methods demand little accuracy in the harmonic amplitudes of the [tidal] potential, since they \textit{use only the frequencies} at which the larger amplitudes appear, and certain details on which to base nodal corrections''\cite{Cartwright:1973em} emphasis added.
Recall that the basis for this analysis is not harmonic in the sense of an orthogonal Fourier series, but rather that it follows from the harmonic development of the \ATGP{} as described in section \ref{sec:basic_potential}.    Here again the historical legacy of tidal terminology lends itself to misunderstanding \citep{Cartwright:2000tt}\citep{Parker:2007wq}.


Harmonic methods of sea level prediction are very successful and will continue to be important for years to come, regardless of other advances.  The conventional and routine harmonic analysis of tide-gauge data provides a foundational set of products that is embedded across the whole coastal economy. Tide tables and derived tidal planes (such as LAT) provide ubiquitous references for both land and marine applications. This situation provides some of the difficulties discussed in Chapter \ref{chp:aggregate}, and is elaborated further in  Chapter \ref{chp:tideFlavours}.


At its foundation, harmonic tidal analysis and prediction characterises the ocean response as a LTI system via a finite set of amplitude and phase terms that are identified empirically by the application of timeseries methods.   This concept is stated a little too simply in the Australian Tide Manual as follows: ``\textit{ the purpose of tidal analysis is to represent the water level or current time series by a set of harmonics, or sine waves, each of them having a specific amplitude and phase}''\citep{PCTMSL-sp9}.
Note the emphasis on periodicity rather than dynamics.  More problematic is the fact that this stated purpose is neither universally agreed nor implemented by any tidal agency, as the following discussion will outline.  
The conventional harmonic tide representation of a timeseries is given in equation \ref{eq:cos}.  At face value involves a linear sum of sinusoids with amplitude $A_k$ and phase lag $g_k$ parameters considered to be constants that characterise the LTI system; other terms and complications are discussed below.  
%---------------------
\begin{equation}
\eta_{tide}(t) = \sum_{k} f_k A_k \cos ( t.\omega_k + g_k + u_k)
\label{eq:cos}
\end{equation}
%---------------------
Significantly, the harmonics involved with such an analysis can only ever be a \emph{subset} of tidal frequencies arising from the decomposition $c(t) = \sum_{k} H_{k} e^{-i( t\theta_{k})}$.  This limitation comes mainly from the non-orthogonality of the basis functions and the fact that all observational records are of finite length.
Conventional tide practice applies a range of elaborations and work-arounds in order to select the set of analysed components and produce predictions.  
Before discussing any further, the essential steps of the harmonic analysis and prediction process are  illustrated schematically in Figure \ref{fig:tidePractice} and summarised as follows:
\begin{enumerate}
    \item formulation of component set based on various conventions;
    \item empirical projection of an observation record onto the resulting non-orthogonal basis functions;
    \item application of conventions accounting for unresolved frequencies; 
    \item synthesis of tidal timeseries by applying matching conventions.
\end{enumerate}

The practical value of harmonic methods reflects the special significance of sea level variation occurring at tidal frequencies. This value does not rely on simulating ocean dynamics. Rather, the very \emph{absence} of fluid dynamics is a key strength of harmonic methods. 
A record of historical observations at one place is essentially all that is required, with no need for spatial information or any additional inputs whatsoever.

For almost all ocean-exposed locations, harmonic analysis can reduce the majority of sea level variance in a long observational timeseries to a remarkably short list of numbers; typically mapping several thousand values to a few dozen \citep{Flinchem:2000kp}.  This unique data compression could be better exploited for on-demand delivery services as suggested in chapter \ref{chp:tideFlavours}. 


The standout power of some tidal variations is apparent in the spectra shown in Figure \ref{fig:obsSpectraEg}. 
But conventional practice has developed additional terms that are not directly apparent in the \ATGP{} that have in some locations greatly improved the value of the resulting predictions; in particular for what are called the compound, shallow water or overtide terms. In this cases, hydrodynamic reasoning has been applied to select and justify additional terms that could result from the behaviour of depth limited waves and wave-wave interactions.
Any seasonal modulation terms are analogous in the sense that the rational for the LTI model has been extended to account for observed patterns rather than derived solely from the \ATGP{}.
%----------------------
\begin{figure}[H]
	\centering
	\begin{subfigure}[t]{\figwidthHalf}
	    \includegraphics[width=\textwidth]{figures/images/zetler_tidal_computer_lady_1921.png}
	    \caption{Harris-Fischer tide machine circ. 1912 \protect{ \citep{Parker:2007wq} }}
    \end{subfigure}
    \hfill
    \begin{subfigure}[t]{\figwidthHalf}
    	\includegraphics[width=\textwidth]{figures/images/DHI_machine_cartwright_fig11p2.png}
	    \caption{Gezeitenrechenmaschine at DHI circ. 1940 \protect{ \citep{Cartwright:2000tt} }}
	\end{subfigure}
	\caption{Analogue tide machines embody harmonic prediction.  A finite set of phase and amplitude values are dialled up and the handle turned to generate a prediction timeseries.}
	\label{fig:tide_machines}
\end{figure}
%----------------------
Harmonic analysis can be formulated as an over-determined inverse problem:
%--------------------
\begin{equation}
    \label{E:Axb}   
    \M{A} \V{x}=\V{b} 
\end{equation}
%--------------------
Where $\M{A}$ is a matrix of tidal timeseries basis functions, $\V{x}$ is the analysed tidal solution and $\V{b}$ is an observational record. The tidal basis functions that comprise $\M{A}$ are developed from the combination of the harmonic decomposition of $\eta_{eq}$ and a series of conventions.  The resulting basis is neither orthogonal nor complete. Inclusion of the modulation terms $f_k$ and $u_k$, discussed below, means that the basis functions are not actually sinusoids. The core assumption of stationarity for the finite series of basis functions is equivalent to an assumption that the basis has infinite span in time.   

Practical applications of the harmonic formalism has largely evolved prior to modern cheap computing, and many details can appear somewhat baroque in isolation.  The analogue instruments shown in Figure \ref{fig:tide_machines} serve to highlight this operational history.


The harmonic idealisation effectively operates on infinite basis functions whereas any observational record is of finite length.
Given the close clustering of tidal lines in frequency-space, equation \ref{E:Axb} is poorly conditioned for inversion.   This matrix condition is further degraded by the presence of non-tidal variations. 
Subsequently, practical harmonic analysis procedures must specify which frequencies to include in the basis set and then somehow account for the effects of unresolved frequencies.

At the time of writing the Bureau of Meteorology process does not systematically account for the inversion quality with regard to signal:noise ratios as suggested by \citet{Foreman:2009bg}.
This fact contributes to the potential for overfitting or projection of non-tidal signal onto standard tide predictions that are relevant to later chapters \ref{chp:aggregate} and \ref{chp:tideFlavours}.


Typically arguments regarding the relative magnitude of the harmonic components of $\eta_{eq}$ are used to prioritise candidate frequencies for inclusion with respect to the observational  record length. 
By convention, the effects of clusters of unresolved tidal frequencies are represented by modulation of the basis functions in relationships assumed to map from $\eta_{eq}$.  
In equation \ref{eq:cos} these modulation terms are included as $f_k$ and $u_k$.  These terms are misleadingly known by convention as `nodal corrections', due historically to the close spectral spacing associated with the the lunar nodal regression; differences in $\omega_k$ of about $\frac{1}{18.6}$ years.     These same terms are more informatively called `satellite modulations' by some authors in recognition that the modulation of a constituent term by a very closely spaced spectral cluster arises from terms other than $d_5$. 

Importantly, this nodal modification of basis functions means that the analysis results do not simply represent amplitudes and phases for regular sinusoids.  


There are other aspects in which the routine production of conventional tide tables deviates from the essential matrix inversion, typically to account for observational records that are fragmentary, short or in some way less than ideal.
For instance when additional constants are inferred or directly copied using assumed relationships from either $\eta_{eq}$ or previous analyses.  


To the extent that the non-astronomical phenomena display predictably periodic characteristics, inclusion within the LTI tide model can be to the benefit of a tide prediction; at least when used for conventional tide tables for instance.   
On the other hand, broadband effects that simply project some colour onto tidal basis functions reflect more about the failure of the fitting approach and should rather be accounted for in the forecast error information. Such error information is generally non-existent in standard tide products and at best is implicitly understood in a qualitative sense by users. 

More basically, it is not clear that users of tide predictions are always cognisant of exactly what phenomena conventional predictions represent.    This issue partly motivates chapter \ref{chp:tideFlavours}.
Consider for now the case of seasonal phenomena.   The harmonic decomposition of $c_{nm}(t)$ provides two relatively tiny signals at periods of 1 and 0.5 years - conventionally termed Sa and Ssa.  Despite the insignificance of seasonal terms in the forcing $\eta_{eq}$, conventional predictions frequently assign  disproportionately large values to account for observed patterns.  
Terms that reflect non-tidal seasonal modulation of semi-diurnal components can also be included in standard predictions;such as H1 and H2 in the \citet{Foreman:1977ua} schedule 
And there are also  diurnal and semidiurnal sea level signals driven directly by meteorology  but coherent with the \ATGP{}. \citet{Ray:2003ui} evaluate \NWP{} representations of \emph{barometric} tides (ie atmospheric tides) at these frequencies with implications for oceanographic processes.  


Treatment of meteorological versus gravitational influences on observed sea level for terms such as Sa, Ssa, S1 and S2 are a point of difference between conventional harmonic analysis and response methods.  
The \underline{response method}, and it's variations, form a distinct branch of academic tidal analysis that stand in contrast to conventional techniques of harmonic analysis.   This consists of a generalised implementation of the LTI model, in which the ocean response to any input sequence is characterised by empirically determined admittance functions using more general timeseries analysis methods.  
\citet{Munk:1966ts} in fact claimed to be motivated by an application of Ockam's razor to the highly evolved historical baggage of harmonic methods.   
It could be said that response methods simply reflect what physical oceanographers have thought of as `tides' for decades.  This is why harmonic constants are now thought of as just one representation of the idea of a tidal admittance. 
\citep[pp 198]{Cartwright:2000tt} provides the following explanation for why the ostensibly modern response \emph{techniques} have not displaced conventional practice: 
\begin{quotation}
    $\dots$ the improvement in predictable variance is numerically small compared with the natural noise in sea level.   Because of this, and the fact that the Response Method is harder for a routine operator to grasp, it has never been adopted for ordinary tide-table production. It remains essentially a research tool for specialists. 
\end{quotation}
But Cartwright's explanation isn't quite the whole story.   
There are benefits for routine production resilience when the product consists of a short list of named parameter values.   The conventions of harmonic constants provide a level of simple intuition that, regardless of prediction skill, are relatively easy to manage in a routine production schedule.
Much more fundamental is the question of the purpose for conducting an tidal analysis at all; and chapter \ref{chp:tideFlavours} argues that delivery of tidal services would be much improved by design choices that do not allow for any ambiguity in this regard.   
Response methods, and its modern extensions such as the orthotide formalism or tidal wavelet techniques, do not target the same outcome as conventional tide prediction.   All may find a useful place in the operational setting as long as relevance for the application is clear. 
Some more on the topic of alternatives to harmonic analysis are addressed in Appendix \ref{appendix:response}, but are not essential to the flow of this discussion.


For emphasis, it restated the harmonic and response models contain the same core concept.   Both model the ocean as an LTI system responding to some type of tidal forcing.   Furthermore, both apply assumptions about the smoothness of ocean admittance functions; in the use of inference in the case of harmonic methods.  Indeed \citep[chpt6]{Fu:2001ub} groups both the standard harmonic and convolution formalism as instances of the `response' formalism on this basis.

By definition these LTI methods represent phenomena stationary in tidal frequency-space.   The value of assuming stationarity is the very basis on which tide tables are built. 
Looking beyond this tide-table view of what actually constitutes an ocean tide, there are a great many instances where non-stationary phenomena are categorised as tidal.    For instance, whenever tidal long waves are modulated by transient or secular changes as described by \citet{Devlin:2017hu};   or even the breaking of internal tidal waves \citep{10.3389/fmars.2021.629372}.
The analysis of non-stationary tides span a broad range of literature and applications \citep{Jay:2003bj}.
But it would seem that a singular focus on the stationary model for tide tables has so far prevented such techniques finding regular application within routine operations for forecasting.  

Arguably, the apparently singular harmonic view of operational tide prediction reflects a historical legacy that places little importance on the distinction between the role of tidal methods for forecasting versus for filtering.

%-----------------%
\subsection{Distinguishing forecasts and filtering}

Despite the prominent role of harmonic analysis in tide prediction, direct application to the production of forecasts is not actually the only reason for employing tidal methods in an operational agency.   
The other is to filter, or de-tide, a signal in order to derive an meaningful non-tidal signal.    Whilst it is possible and often convenient to de-tide observed sea level by simply subtracting a standard tide prediction, this is not always fit-for-purpose.
It is consistent to de-tide observations with a standard prediction when the objective is to evaluate the error of the tide prediction itself; conventionally termed a 'tidal residual'.
But when the objective is to remove the variability directly associated with the \ATGP{}, or alternatively to optimally reduce the tidal frequency variability within a sample, then the standard tide prediction is likely not a good choice.

Operational oceanography in the form of \BL{} relies on spatial tide models to de-tide (``correct'') sea level observations from altimeter in order to assimilate the resulting sea level anomaly quantity (SLA).   This de-tiding process has been designed to render the observations compatible with the physics of the model.  But ensuring compatibility is not trivial and  many viable solutions are in production via sources such as RADS \citep[table 3.2]{Scharroo:2011vd}.  The details of how the altimeter tidal correction is constructed may have implications for interpretation of the final ocean forecast; long-period tides \citep{Egbert:2003jd} and nongravitational tides \citep{Arbic:2005gv} are highlighted as special points of interest with regard to operational sea level forecasts.

As noted above, the current configuration of \BL{} does not assimilate tide gauge data, though in principle these insitu observations could be reduced to a consistent SLA quantity via appropriate de-tiding; for example as in \cite{Matsumoto:2000tg}.

Figure \ref{fig:residualExample} illustrates how observations de-tided for different purposes produce roughly similar but certainly not identical anomaly signals. 

%-------------------------
\begin{figure}[h]
\begin{center}
    \includegraphics[width=\figwidthFull]{figures/plots/diag_plot_014406_detide_compare_20120101.png}
    \caption{Illustration of de-tiding observations by subtraction of tide predictions.  Different `residuals' result from alternative tide predictions: [1] regional gridded tide solution, [2] harmonic prediction and [3] harmonic prediction with significant non-gravitional harmonics removed. }
    \end{center}
    \label{fig:residualExample}
\end{figure}
%-------------------------

%-----------------%
\subsection{Global tides}
\label{sec:spatialTides}
Extending the basic premise of tidal analysis to the whole ocean results in a global tidal atlas.  A tide atlas, or global tide solution characterise the tidal admittance across the whole ocean surface; rather than just a single point.
Global tide solutions have been enabled by satellite altimetry.
Indeed, the design and motivation for satellite altimetry missions have been to a large degree driven driven by the study of ocean tides.
Meaningful tidal atlases existed prior to altimetry, most significantly that due to Schwiderski \citep{Schwiderski:1983ke}, but these necessarily suffered from a lack of validation and constraint in the deep ocean.


Global tide models are employed for a range of purposes beyond sea level forecasting,often as an intermediate step or a correction, for applications in gravity, orbit determination, earth rotation and even the definition of coordinate systems \citep{Anonymous:2004tm}.

Most importantly for the present discussion, global tide models are employed to \underline{filter}, or de-tide, sea level observations from near-real-time altimetry as a data assimilation constraint on mesoscale ocean forecast modells; discussed in section \ref{sec:mesoscaleOperational}


The typical visualisation of a tidal atlas is in the form of maps of time-invariant tidal amplitude and phase for a sequence of named constituents; regardless of whether a harmonic method was actually employed to produce the solution.   Each map describes a component wave or ``partial tide''.  These are maps are also called `co-tidal' and `co-range' plots.
In line with the discussion in Section \ref{sec:LTI}, whole ocean is conceived as a LTI system. 
The long spatial scale of these component waves and the existence of spatial nodes places special importance upon amphidromes or amphidromic points - nomenclature introduced by Harris in the late 19th century \cite[pp 119]{Cartwright:2000tt}.  Existence and placement of amphidromic points is an important visual metric employed to assess tidal model results.  
Figure \ref{fig:atlas} shows a the typical tidal atlas map for a single constituent with amphidromic systems apparent as radial patterns in the co-phase diagram.

Modern tidal atlases are in close agreement with regard to broad patterns, but characteristically differ at the shallow water margins; the very location of most direct interest to sea level forecasts.  Figure \ref{fig:tpx_cross} illustrates the fact that modern global tide models typically agree within about 0.02m in the deep ocean, but can differ substantially in coastal and shelf regions.  
%---------------------------
\begin{figure}[h]
    \begin{center}
    \includegraphics[width=\figwidthBig]{figures/maps/global_m2_tpx08.png}
    \caption{Example tidal altas showing cophase and corange diagrams for a single tidal component M2.  Data source TPX08 \cite{Egbert:2002ug}  }
    \end{center}
    \label{fig:atlas}
\end{figure}
%---------------------------
\begin{figure}[h]
    \begin{center}
    \includegraphics[width=\figwidthBig]{figures/maps/map_tide_differences_tpx_xovers.png}
    \caption{Maximum difference between tidal timeseries for 2012 at topex cross-overs near Australia.  The different solutions agree very closely in deep water, whilst the significance of shallow water effects are apparent.  Models included: CSR04\citep{Eanes:1996tr}, FES04\citep{Lyard:2006ir}, DTU10\citep{IMPROVEMENTOFGLOBA:2010tu}, GOT47, GOT48\citep{Schrama:1994vr}\citep{Ray:1999vm} }
    \end{center}
    \label{fig:tpx_cross}
\end{figure}
%---------------------------
%TBC 
%Altimetry has directly enabled contrasting scientific developments in global tides (eg.\citet{Egbert:1996vr},  \citet{Lefevre:2011dg}) and mesoscale ocean variability (eg.  \citet{Wunsch:1998bq}, \citet{Chelton:vi}).
The LTI concept embodied in tidal atlases is well suited to representing deep water tides.   Amplitudes of each partial tide are no larger than about 0.02 cm for the majority of the global ocean, compared to depths of around 4km.
Given the underlying LTI framework, the tidal literature has naturally focused on stationary frequency-space metrics.  Intercomparison of mode1s and assessment against observations is almost always performed at a small set of dominant tidal frequencies - commonly only M2, K1, O1 and S2.

Summaries that categorise the many global tide models on the basis of design choices, parameterisations and data assimilation methods are given in \cite{Ardalan:2008gs} and \cite{Matsumoto:2000tg}. 

Tidal atlases are in essence no different from one-dimensional tide predictions.    Discussions above on the topic of formalisms apply equally to global atlases as to insitu timeseries.  
And similarly to insitu tidal products, special attention is warranted regarding the treatment of signals associated with non-gravitational effects and the possible differences between a forecast model and an correction/filter.



Some authors have gone so far to presented the existence of repeat-orbit altimetry observations as effectively being `thousands of tide gauges', but reduction of these observations via tidal analysis requires many special considerations. Most obviously is that ground-path of these orbits results in repeated but infrequent samples of any fixed point; and subsequently frequency aliasing is a primary limitation on tidal analysis methods..



Despite the general equivalence, a point of practical difference is the treatment of the non-linear effects significant in shallow water.
In contrast to the apparent convergence of surface tide models in the open ocean, shelf and coastal regions are problematic.   In shallow waters wavelengths shorten and nonlinear interactions between partial tides can become very prominent. 


A strength of conventional 1-dimension analyses of coastal tide gauges has been the incorporation of shallow water compound tides.  It is not atypical in Australian locations to include dozens of nonlinear frequencies in a harmonic analyses.   Nonlinear signals observed at a tide gauge are often attributed to complex very localised dynamics.  
Understandably, global tidal atlases in optimising the linear deep water signal and have relatively poorly represented or ignored coastal nonlinearties.  The nonlinear M4 signal (associated with self-interaction of M2 waves) is perhaps the only such partial tide to be included in many modern atlases and is observable with altimetry \cite{Ray:2010jm}.


As a forecast product, tidal atlases have nothing like the broad economic integration of conventional coastal tide tables.  The Australian Bureau of Meteorology does not promulgate official tide predictions away from insitu observation locations. 
For the contemporary operation setting, tidal atlases are primarily relevant as an intermediate product to enable satellite observations and to apply as open boundary condition forcing to downscaled simulations.


Coordinated development of improved tidal atlases with specific performance requirements formed a significant component of altimetry missions in the 1990s.
\begin{quotation}
This international effort quickly split into two main approaches: the so-called empirical approach based on the direct analysis of the altimetry sea level time series \dots{}, and a modelling approach based on hydrodynamic and assimilation models. Later on, the interaction between the two approaches (i.e. data assimilation based on altimetry analysis on one hand, and hydrodynamic/assimilation modelling on the other hand) was a key factor for the overall success in improving tidal prediction accuracy and reaching the T/P requirements \cite[pp394]{Lefevre:2011dg}.
\end{quotation}

No global tidal atlas can be empirical to the extent of simple timeseries analysis - the spatial and temporal coverage of observations are too sparse.

Schwiderski's pre-altimetry solutions relied heavily on global compilations of harmonic constants for mainly coastal tide gauges, from which spatial maps were created via a `hydrodynamic interpolation' method that would in hindsight be considered an application of data assimilation \cite[pp822]{Egbert:1994wz}.

Now with over 20 years of altimetry data, all the highly evolved global tidal models or tidal solutions employ \underline{data assimilation} in some manner.  That is, they make some combined use of dynamic models and observational data.  
The data assimilation employed for global tide solutions has quite a different form to that used in forward models like \BL{}; section \ref{sec:mesoscaleOperational}.  Whereas a mesoscale ocean model requires initial conditions from which to integrate forward in time, the LTI tidal conception of the ocean seeks a single optimal solution in frequency space. 


The dynamics relevant to tidal sea level are conventionally written as the Laplace Tidal Equations (LTE) ; \cite[9.8]{gill1982atmosphere} and \cite{Hendershott:1981ub}.  The LTE are a set of depth integrated shallow water equations on a rotating thin shell.  Advection is neglected altogether.
The combination of barotropic hydrodynamics and data assimilation used in global tide models has proven to be a adequate when the aim is to map tidal patterns of surface elevation.  The LTE provide a tractable means of doing so given the incomplete spatial and temporal coverage of observations.
This formulation excludes a direct representation of internal tides, despite the important role they play in the energetics of ocean tides.
\begin{quote}
    Barotropic tides generate internal tides, and internal tides in turn feed back onto the barotropic tides. Inferences from altimetry-constrained barotropic tide models show that about one-third of global tidal energy dissipation occurs in regions of rough topography, where internal tides are generated \dots{}. Internal tide generation thus acts as a damping mechanism for the barotropic tides.\citep[pp22]{Arbic:hy}
\end{quote}
Depth integrated barotropic LTE relegate the effect of internal mechanisms on surface elevation to parameterisations.  The dissipative stress term ($F$ in eq \label{E:LTE_momtm}) stands for all losses of energy from the barotropic simulation.  Given the reality of ocean stratification, this loss term accounts for the conversion of barotropic to higher baroclinic modes in any global solution \cite[pp121] {gill1982atmosphere}.
Internal waves at tidal frequencies do have an observable surface signature albeit relatively small \cite{Ray:2011tj}.   
And for sea level forecasting, internal modes are ostensibly of interest insofar as they impact the prediction of surface elevation.  
Internal tides are in general not stationary and not as predictable as surface tides \cite{nash2012}.  The extent to which aperiodic internal modes can be separated from surface tides is a measure of the value core stationarity model of  tide prediction.
The tidal view of the ocean as a LTI system driven by the \ATGP{} is very useful, but with the caveat that ``the ocean is a physically complex and noisy filter.  In consequence, tidal harmonics are not strictly constant \citep[197]{Ray:2010jm}''.  
Whilst stratification and all internal hydrodynamic mechanisms are very coarsely parameterised in global tide models, the internal structure of the ocean is a primary focus of \OGCM{}s, and this forecasting perspective is discusssed in section \ref{sec:mesoscaleOceangraphy}.  Some further detail on on modeling global tides is also included in Appendix \ref{appendix:globalTides}.

%-----------------%
\subsection{Operations needs more than one ocean tide}
The above sections have portrayed the ways that ocean tides have come to be employed for operational forecasts.
In the simplest terms, there are two operational conceptions of ocean tides actually are;
\begin{itemize}
    \item tides as a predictable periodic variation in total water height and;
    \item tides as a longwave physical response to gravitational forces.
\end{itemize}
Whereas methods for representing and forecasting both of these are centered on the linear-time-invariant concept of an ocean admittance, they are not identical.
Furthermore, the aim of forecasting the total water height differs greatly in intent from the aim of filtering or decomposing water height; and there is no reason to require both to be satisfied by a single conceptualisation of an ocean tide. 
Allowing for ocean tides to be other than singular may give rise to some unique complications given that conventional tide predictions have come to play such an established role in coastal economies.   A notable role in this regard is how derivations from conventional tide predictions are built into spatial references such as nautical chart datum and highest astronomical tide (HAT).   Ensuring that forecast products are consistent with these spatial applications would appear to require some care.  
Most broadly, there is room in the operational setting for improved clarity in handling tides in forecasting and this topic is addressed in both chapters 
\ref{chp:aggregate} and \ref{chp:tideFlavours}. 

%-----------------%
\section{Sea Level Forecasts at the Overlap}
\label{S:fc_prospects}

Whatever the promise of inclusion of explicit tides within \OGCM{}s scientifically, application to operational sea level forecasts is not well explored or justified.
Operational forecast systems must balance a unique set of motivations quite distinct from other \OGCM{}-based simulations.  \\
Nested models of increasing resolution may be appear the obvious direction for coastal sea level forecasting.  But in contrast to the status of \BL{}, operational realisation of such a capability is not imminent nor guaranteed.


Two broad directions are identified with regard to the prospects for \OGCM{}-based operational sea level forecasts:
\begin{enumerate}
\item \underline{Aggregation}: the \OGCM{} is used to forecast a nontidal quantity that is subsequently aggregated with tide predictions and other forecast products;
\item \underline{Integrated simulations}: the \OGCM{} is used to forecast as much of the sea level signal as is feasible and is used directly.
\end{enumerate}
These equally apply to the realistic scenario of employing a spatially nested set of simulations in that \obc{}s may be provided by either aggregation or integration.\\


Some form of aggregation does appear most promising for \emph{routine} forecasts in the current operational setting.  
This assertion is based on consideration of the highly evolved specialisations of forecasting methods as well as organisational realities of system development.\\
Caveats aside, for periodic signals direct output from a dynamic model is unlikely to approach the ability of harmonic methods to predict complex local and nonlinear effects.   The fundamentals of conventional tide predictions reasonably provide the reference for sea level forecasts more generally. \\



More inclusive physical simulations and aggregation are not however mutually exclusive.  
And one promising concept involves aggregation \emph{and} a tide-resolving \OGCM{}.  The premise is that an \OGCM{} will only ever represent a compromised tidal signal, but inclusion of these dynamics will produce a net gain for the nontidal flow.\\
An operational reference for such an approach is found in the design of the UK Storm Tide Warning Service (now titled UK Coastal Monitoring and Forecasting) \cite{Horsburg:2009ui}.   Although the UK system is based on depth integrated surge models, rather than an \OGCM{}, the aggregation of hydrodynamic models and tide predictions is instructive.\\   








