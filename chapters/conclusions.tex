
\chapter{Concluding discussion}\label{c:conclusions}

This body of work was motivated by specific conceptual problems identified beneath the forecasting of sea level at the Australian Bureau of Meteorology.  The intersection of conventional tide prediction and operational mesoscale ocean forecasts within the agency is problematic with regard to issues of representation, system compatibility and user expectation.
This desktop study has intentionally not pursued higher resolution simulations, but instead directed attention to some less-studied aspects of evaluating and exploiting established operational systems.   Such as perspective is relevant to the planned development of multi-model seamless services.   
%\emph{%
%Is the operationalisation of high resolution coastal simulations a prerequisite for providing regular sea level forecasts?  Despite comparatively low coastal resolution, can \BL{} directly provide useful prognosis of non-tidal coastal sea level and with what limitations? }
%\emph{Are point-by-point tide-gauge observations the only relevant basis on which to compare and assess heterogeneous sea level forecast simulations? 
%Why is the role of coastally trapped waves prominent in the oceanographic literature but seemingly absent from operational forecasting?}
%\emph{Has numerical simulation rendered conventional tide prediction redundant for operations?  How can the unique traditional function of tide prediction compliment nominally non-tidal ocean forecasts and future seamless prediction systems? }
Much of the body of this thesis has been published in a series of peer-reviewed articles.
%------------------------
\section{Findings summary}
In answer to the research questions posed in \ref{Sec:intro} the following points summarise the thesis findings:
%-------------
% external file

\begin{itemize}
    \item Incompatible definitions of ocean ``tide'' are in parallel operational use;
    \item despite spatial scale limitations, mesoscale ocean forecasts can directly provide significant forecast value for coastal sea level, but with many caveats;
    \item nominally tidal signals are present in mesoscale non-tidal ocean simulations and require care to avoid misinterpretation; 
    \item a simple aggregation approach to existing heterogeneous data provides an important skill benchmark for future sea level forecast system development; 
    \item point-based bias correction characteristics indicates that coastally contiguous extensions to model aggregation are feasible;
    \item the coastal propagation characteristics of candidate forecast systems can be qualitatively evaluated and compared in a grid-independent waveguide projection; 
    \item a coastal waveguide projection offers a means to direct forecaster attention to signals of special relevance along the Australian mainland coast;
    \item conventional harmonic tide predictions are not redundant but require appropriate product differentiation to compliment modern applications and facilitate future refinement.
\end{itemize}
%-------------

%------------------------
\section{Heterogeneous simulations, compatibility and seamless services}

TBC

Discussion about why ...and future 



