Firstly, it was demonstrated that incompatible definitions of ocean ``tide'' are in parallel operational use.
Whereas downscaling for coastal sea level forecasts is clearly a productive approach, mesoscale ocean forecasts can immediately and directly provide significant but qualified forecast value for coastal sea level.
The fact that nominally tidal signals are present in mesoscale non-tidal ocean simulations means that care is required to avoid misinterpretation.

An aggregation approach that combines existing heterogeneous data but accounts for double-counting provides an important skill benchmark for future sea level forecast system development.
The point-based bias correction characteristics from these aggregated forecasts indicate that coastally contiguous extensions to model aggregation may be feasible.


In the operational context of combining and upgrading forecast models, it was shown that the coastal propagation characteristics of candidate forecast systems can be usefully evaluated and compared in a grid-independent waveguide projection.
Such a coastal waveguide projection also offers a means to direct forecaster attention to signals of special relevance along the Australian mainland coast.

Finally, it was argued that conventional harmonic tide predictions are not redundant, despite the ongoing advances in hydrodynamic simulation,  but that operational tide services require appropriate product differentiation to compliment modern applications and facilitate future refinement.
