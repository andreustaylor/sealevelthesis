%-----------------%
\section{Sea Level Forecasts at the Overlap}
\label{S:fc_prospects}


\subsection{Towards seamless forecasts from evolving systems}

Whatever the promise of inclusion of explicit tides within \OGCM{}s scientifically, application to operational sea level forecasts is not well explored or justified.
Operational forecast systems must balance a unique set of motivations quite distinct from other \OGCM{}-based simulations.  \\
Nested models of increasing resolution may be appear the obvious direction for coastal sea level forecasting.  But in contrast to the status of \BL{}, operational realisation of such a capability is not imminent nor guaranteed.


Two broad directions are identified with regard to the prospects for \OGCM{}-based operational sea level forecasts:
\begin{enumerate}
\item \underline{Aggregation}: the \OGCM{} is used to forecast a nontidal quantity that is subsequently aggregated with tide predictions and other forecast products;
\item \underline{Integrated simulations}: the \OGCM{} is used to forecast as much of the sea level signal as is feasible and is used directly.
\end{enumerate}
These equally apply to the realistic scenario of employing a spatially nested set of simulations in that \obc{}s may be provided by either aggregation or integration.\\


Some form of aggregation does appear most promising for \emph{routine} forecasts in the current operational setting.  
This assertion is based on consideration of the highly evolved specialisations of forecasting methods as well as organisational realities of system development.\\
Caveats aside, for periodic signals direct output from a dynamic model is unlikely to approach the ability of harmonic methods to predict complex local and nonlinear effects.   The fundamentals of conventional tide predictions reasonably provide the reference for sea level forecasts more generally. \\



More inclusive physical simulations and aggregation are not however mutually exclusive.  
And one promising concept involves aggregation \emph{and} a tide-resolving \OGCM{}.  The premise is that an \OGCM{} will only ever represent a compromised tidal signal, but inclusion of these dynamics will produce a net gain for the nontidal flow.\\
An operational reference for such an approach is found in the design of the UK Storm Tide Warning Service (now titled UK Coastal Monitoring and Forecasting) \cite{Horsburg:2009ui}.   Although the UK system is based on depth integrated surge models, rather than an \OGCM{}, the aggregation of hydrodynamic models and tide predictions is instructive.\\   

\subsection{Handling real observations}
Operational relevance of mixed-quality tide gauges.

The evolving national network - distinguished from uniform or curated climate data.


Datums!

Tide gauges and altimeters.    


Data driven and ML


\subsection{Who's skill metric}
Assessing forecast goodness

tidal "constants"

timeseries

bias

Special role for "official" promulgated tides - 


\subsection{Unresolved processes and model-data fusion (DA)}
No forward model is useful by itself.

Initial conditions and/or boundary conditions 

Application of DA not unlike the training data in ML ...it all depends on what the target is.



\subsection{Predictability benchmarks}

Conventional tides as a benchmark and reference ...more like a literal benchmark


Existing operational forecast system as benchmark for future developments.


Peterson and motivation toward conreteness.





