
% Optional preface to the dissertation
\begin{preface}

\noindent
This thesis is built around a series of journal publications. \\
Chapter \ref{chp:aggregate} reproduces material originally published as:

\vspace{5mm}
%\hangindent=3em
%\hangafter=1
\noindent
\texttt{Taylor, A. J., and G. Brassington, 2017: Sea Level Forecasts \\ Aggregated from Established Operational Systems. Journal of Marine Science and Engineering, 5, 33, \url{https://doi.org/10.3390/jmse5030033.}}


\vspace{5mm}
\noindent
Chapter \ref{chp:waveguide} reproduces material originally published as:

\vspace{5mm}
%\hangindent=2em
%\hangafter=1
\noindent
\texttt{Taylor, A. J., and G. B. Brassington, 2020: Sea Level Anomaly\\ Forecasts on a Coastal Waveguide. Weather and Forecasting, 35,\\ 757$\-$770, \url{https://doi.org/10.1175/waf-d-18-0198.1.}}

\vspace{5mm}
\noindent
Extension material from the later was presented as:

\vspace{5mm}
%\hangindent=2em
%\hangafter=1
\noindent
\texttt{Taylor, A., D. Greenslade, X. Zhou, and G. Brassington, 2020:\\ National nontidal sea level forecasts on a coastal waveguide.\\ Coast Eng Proc, https://doi.org/10.9753/icce.v36v.currents.26.}

\vspace{5mm}
\noindent
The material comprising chapter \ref{chp:tideFlavours} was in preparation for stand-alone publication at the time that this thesis was submitted.

\vspace{5mm}
\noindent 
All funding and in-kind support for this research was provided by the Australian Bureau of Meteorology.

\end{preface}

