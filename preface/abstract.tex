
\begin{abstract}
This thesis with publication addresses conceptual issues at the intersection of sea level forecasting, tides and mesoscale oceanography with implications for operational practice.
New methods for combining, evaluating and delivering nominally ``tidal'' sea level information are proposed and analysed; with a focus on representation issues and system compatibility.
The results are expected to inform the development of seamless forecasting services and assert that aspects of conventional tide prediction will maintain relevance by supporting increasingly sophisticated numerical and data-driven prognostic tools.



The study motivation and scope originates from within the ongoing operational movement towards what are increasingly called seamless services. 
All of the data and systems treated in this study reflect the setting within the Australian Bureau of Meteorology; though the findings are not specifically restricted in relevance to Australia.



The document is structured as follows; \\
Chapter \ref{chp:introduction} portrays the operational context and establishes the relevance of the study scope being restricted to the overlap of mesoscale forecasts and tide prediction;  
chapter \ref{chp:forecastConcepts} unpacks relevant technical details and concepts within the seamless framework to highlight problematic areas of overlap or incompatibility; 
chapter \ref{chp:aggregate} responds to this situation of overlap to present a methodology for deriving improved forecast value from existing systems to both set a performance benchmark for candidate ocean forecast system updates and elucidate details of predictability relevant to operational services;
chapter \ref{chp:waveguide} also addresses the topic of updates and extensions to the operational ocean forecasting suites by presenting an approach to connect the academic literature treating coastally trapped waves with the operational evaluation of candidate systems and guidance provided to forecasting staff;
chapter \ref{chp:tideFlavours} returns to the role of conventional tide prediction within the evolving operational suite of forecasting systems and proposes incremental but significant changes to the nature of that service to mitigate issues related to conceptual overlap and allow for ongoing ;
finally chapter \ref{chp:conclusions} considers the results of the study in the context of the movement towards seamless forecasting to discuss how the subject of sea level and tide prediction presents a case for including more than just the tidy chain-of-scales in plans for forecasting services.



The primary findings of the thesis are summarised as follows:
Firstly, it was demonstrated that incompatible definitions of ocean ``tide'' are in parallel operational use.
Whereas downscaling for coastal sea level forecasts is clearly a productive approach, mesoscale ocean forecasts can immediately and directly provide significant but qualified forecast value for coastal sea level.
The fact that nominally tidal signals are present in mesoscale non-tidal ocean simulations means that care is required to avoid misinterpretation.

An aggregation approach that combines existing heterogeneous data but accounts for double-counting provides an important skill benchmark for future sea level forecast system development.
The point-based bias correction characteristics from these aggregated forecasts indicate that coastally contiguous extensions to model aggregation may be feasible.


In the operational context of combining and upgrading forecast models, it was shown that the coastal propagation characteristics of candidate forecast systems can be usefully evaluated and compared in a grid-independent waveguide projection.
Such a coastal waveguide projection also offers a means to direct forecaster attention to signals of special relevance along the Australian mainland coast.

Finally, it was argued that conventional harmonic tide predictions are not redundant, despite the ongoing advances in hydrodynamic simulation,  but that operational tide services require appropriate product differentiation to compliment modern applications and facilitate future refinement.


\end{abstract}
\clearpage
