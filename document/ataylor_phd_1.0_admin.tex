\newpage
%---------------------------------------------------------------------------------------------------------------------------------------------------------------
%Section ----------------------------------------------------------------------------------------------------------------------------------------------------
\section{General Background}  \label{S:a} 

%--------------------------------------------------
\subsection{Executive Summary}
\texttt{\TITLE{}}\\

This study concerns the intersection of two heterogeneous approaches to forecasting sea level - tidal harmonics and operational oceanography.\\



Operational oceanography in the \GODAE{} heritage is a relatively new arrival into forecasting centres such as the \BOM{}. \OGCM{}-based systems including \BL{} now provide regular ocean state forecasts with full-regional coverage.  These are distinct from localised and event-only ocean models that have a longer history in relation to tropical storm surges.\\
Opportunities and problems arising from this new situation invite further investigation, including a reconsideration of some well-embedded aspects of harmonic tide prediction.\\


The Review included in this document demonstrates a critical understanding and proficiency with the literature of these two sub-disciplines and details the context from which the original research launches.\\
Subsequently, the work plan outlines a sequence of distinct problems relating to understanding and exploiting the interplay of operational \OGCM{}s and tidal analysis with regard to forecasts over timescales of hours to days.

An alternative wording of the title that may better emphasise the science content of the work is as follows:  \texttt{\ALTTITLE{}}\\


%--------------------------------------------------
\subsection{Administration}
Enrolment in this part-time Phd program commenced March 2011. \\


Mr Taylor is concurrently employed as a research scientist with the CAWCR Ocean Observation, Assessment and Prediction program based at 700 Collins st Melbourne.  The study is enabled by the Bureau of Meteorology \emph{studybank} program. No material support or office space is provided by the University.  \\
The enrolment is intentional related to Mr Taylor's work activities and is undertaken with an eye towards formalising and extending the science beyond the scope of the Bureau's projects.  

\BoxBegin
The primary supervisor of the candidature is Dr Gary Brassington. \\
Dr Brassington is also the research leader of Mr Taylor's work within CAWCR.
Dr Kevin Walsh is co-suprvisor and direct contact with the School of Earth Sciences.
\BoxEnd

%--------------------------------------------------
\subsection{Justification for research effort}
For formal confirmation of candidature, the University requires a concise justification in terms of relevance and importance.

Why is this research \underline{relevant?}	
\begin{itemize}
\item Quantitative forecasts of sea level provide environmental intelligence to a broad range of public and commercial stakeholders.
\item Recent developments have altered the context in which operational sea level forecasts are produced;
\item Improvements to forecasts of sea level appear possible but raise unresolved research questions. 
\end{itemize}

Why is this research \underline{important?}
\begin{itemize}
\item Aspects of contemporary sea level forecasts show deficiencies and inconsistencies that require resolution.
\item Perceived opportunities require a research effort beyond the present scope of the \BOM.
\item Sea level forecasts influence decisions of high economic value and social importance: including safety of human life, protection of property and efficiency of coastal operations.   
\item Australia's coastal zone is increasingly important due to population movements and economic development.
\item Neglecting to address this research risks under-utilisation of costly public services and/or the provision of compromised forecast products.
\end{itemize}


%--------------------------------------------------
\subsection{Additional notes regarding activity} 
Work towards this Phd program is being carried out part-time in parallel with related research work at CAWCR.  There is considerable overlap with regard to developing technical skills and knowledge of the literature. \\
Whilst this thesis extends beyond the scope of the employment requirements and project plans, the outcomes of will however be relevant to CAWCR in the provision of guidance and assessments of prospects for future developments.\\

\subsubsection{Engagement with the research community}
Enrolment in the Phd program has facilitated and motivated additional engagement with the research community; beyond the requirements of employment.  Particular instances that are considered illustrative of this aspect of study are listed below.


Regular participation in \underline{local seminar programs}.\\
Most frequently attended are the regular research seminars at CAWCR that typically involve visiting scientists from meteorological and oceanographic institutions.\\
When feasible, it has been relevant to attend oceanographic seminars hosted by Alex Babanin at the Swinburne University Centre for Ocean Engineering, Science and Technology.  It is noteworthy that these seminars often bring together a wider oceanographic community that includes the commercial sector.\\
Although not based on campus, it has possible to attend several postgraduate talks by fellow students within the Atmosphere and Oceans Group of Melbourne University School of Earth Sciences.\\


The following lists a selection of relevant \underline{Australian and International meetings.} attended:
\begin{itemize}
\item Dec 2011 MODSIM conference in Perth.  Oral presentation: `Ocean meets River at the Bureau: Connecting ocean forecasts and river height predictions for improved flood warnings'  \citep{Taylor:2011ud}\\
\item Jan 2012 AMOS conference in Sydney.  Poster: `Coastal sea level forecast evaluation OceanMAPSv2.0'. \\
\item Sep 2012 ESA meetings in Italy:
\par `20 Years of Progress in Radar Altimetry Symposium'
\par `6th Coastal Altimetry Workshop'
\par These two events were a valuable exposure to the international community of researchers directly involved with satellite altimetry and sea level more generally\\
\item Jun 2013 Asia Oceania Geosciences Society Meeting in Brisbane.  Oral presentation: `Representation of Coastal Propagating Sea Level Signal in OceanMAPS Analysis: Case study using NSW tide gauges'.
\item Aug 2013 Permanent Committee on Tides and Mean Sea Level meeting in Hobart.  Oral presentation regarding aggregation of sea level forecasts.
\end{itemize}



Worthy of special mention, an \underline{international study trip} was undertaken in May 2012. Comprising a sequence of short visits to leading tidal experts at institutions located in USA and Canada. This trip was funded by CAWCR.   In contrast to a conference setting, this opportunity was particularly valuable in initiating and facilitating one-on-one dialogue with researchers directly relevant to this thesis.\\
Four separate institutions were visited for 1-2 days each.  A seminar talk related to \BL{} was also delivered at each location.    
\begin{itemize}
\item Department of Fisheries and Oceans - Institute for Oceanographic Sciences located on Vancouver Island,BC, Canada. \\
Primary contact: Dr Mike Foreman.
\item Portland State University, Portland Oregon, USA. \\
Primary contact:Dr Ed Zaron.
\item Oregon State University, Corvalis Oregon, USA.\\
Primary contact: Dr Gary Egbert.
\item University of Hawaii Manoa, USA.\\
Primary contact: Dr Mark Merrifield.
\end{itemize}



\subsubsection{Writing skills}
It has been relevant to invest some study effort into improving writing skills and understanding the process of academic research.  In this respect `The Craft of Research' book by \citep{Booth:2009vy} has been found particularly informative. \\


Similarly, some effort has been invested in developing proficiency with the Latex platform of document preparation.  This path has been taken following many discussions with colleagues about the relative advantages of alternative methods.


\subsubsection{Mathematical methods}
Linear algebraic methods and concepts a prominent across literature of tides, data assimilation and dynamic modelling.   Proficiency with these methods and concepts was identified early in the enrolment as an important requirement, and subsequently significant study effort has been invested in fundamentals.  The writings and lectures of Gilbert Strang available via the MIT open courseware resource have been valuable to this end. \citep{Strang:2010tk}


\subsubsection{Data and tools}
Complementing the literature review, some effort has been invested into gaining basic proficiency with certain datasets and software tools.  The time expended on such explorations is noteworthy given the typically customised and unique nature of each item. For instance, the various global tide models inspected are effectively mutually exclusive with regard to compilation and formats.\\
Similarly, development of the \ATGP{} directly from astronomical ephemeris served both to prepare a fundamental research dataset and as a enabling study exercise - but also requiring an investment of time into understanding and preparing software tools.





