%Section ----------------------------------------------------------------------------------------------------------------------------------------------------
\newpage
\section{Theory: Phenomena and observations}  \label{S:THEORY_OBS} 

%--------------------------------------------------
 %--------------------------------------------------
 \subsection{}
 %--------------------------------------------------
 %--------------------------------------------------
 
Finally, it is relevant to respond to the literature covering sea level \textbf{observations} and certain regionally significant \textbf{phenomena}.\\
A very general and fundamental limitation is that the ocean is under-observed.  Current technology and asset deployment is such that sea level is measured very sparsely in time and space. However, the availability of continual near realtime telecommunication from coastal tide gauges and satellite altimetry has enabled a practical role for data assimilation.  Ocean data assimilation methods attempt to objectively combine information from sparse observations with modelled dynamics to best estimate ocean state.\\
Satellite radar altimeter missions are of particular importance, but bring many complications.  The derivation of a meaningful sea level quantity from these remote platforms is dependant upon a plethora of corrections and choices.  The appropriate use of these measurements is context dependant and an active field of research.\\
Whilst tide gauges and altimeters nominally measure sea level, the connection between the two is problematic.  Most simply, the reduction of observations to a common vertical datum is difficult.  Other relevant details include the role of earth tides, the land/sea interface and topographic length scales.  Despite the requirement for detailed technical understanding, these routine observations are fundamental to progressing sea level prediction. \\
It is noted that the planned \texttt{SWOT} wide swath altimeter mission is the only known prospect for radical expansion of sea level observability and may occur in 5-10 years \citep{OSTST:2012tg}.\\


Some coastal and shelf phenomena are of particular relevance to the time and length scales addressed by this study.  The regional oceanography of these phenomena informs the interpretation of the observations and forecasts.\\
Coastally trapped wave phenomena have proven to be of special relevance along Australia's southern shelves ...
Mesoscale eddies ...\\
Seasonal quasi-periodic phenomena such as the coastal current systems.\\
Internal tides ...\\
River outflow ...\\

Examples of especially relevant publications with regard to this aspect of the research are as follows:\\
\citep{Fu:2001ub}\\
\citep{leBlond:2009tl}\\
\citep{Brink:1991dl}\\

%%REF Horsburg - coastal surge and corrections\\
%%REF Church, Hannon etc CTW\\




