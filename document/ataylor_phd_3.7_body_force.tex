\newpage
\subsection{Sensitivity to tidal body forcing implementations in an \OGCM{}}
\label{S:plan_bodyforcing}


\subsubsection{Problem and motivation}

The prospect of explicitly resolving ocean tides within an operational \OGCM{}s raises many issues of numerical implementation. For this study, the treatment of the tidal body forcing is investigated.\\  
Practical formulations of the body forcing require some compromise - discussed in Section \ref{S:ATGP_extras}).   It is not unreasonable to expect that the simulated ocean response could show sensitivity to the details of this formulation; perhaps especially so given the significance of resonance phenomena to ocean tides \citep{Muller:2008hs}.\\



At face value the tidal body forcing implemented in \MOM{} appears overly simplistic [Section \ref{S:numerical_impl}].  However, the practical significance of body forcing nuances are not well understood.  \\
The following aspects of body force implementation within \MOM{} are raised:
\begin{inparaenum}[(a)]
\item order and degree of spatial spherical harmonics;
\item formulation of temporal variation of $c_{nm}(t)$ (harmonic versus direct);
\item treatment of SAL
\item opportunities for forcing adjustments derived from data assimilative tide models.
\end{inparaenum}

Techniques and results associated with global tidal atlases are pertinent to addressing these topics.\\



In addressing the significance of body forcing formulations, the following research questions are posed:
\begin{itemize}
\item How does a contemporary pre-computed tidal SAL field compare to the scalar approximation implemented in \MOM{}?
\item How does $\eta_{eq}$ field differ when derived directly from numerical ephemerides compared to approximation implemented in \MOM{}?
\item Is implementation of pre-calculated tidal forcing fields into \MOM{} feasible within the split-explicit configuration?
\item Can a time-domain force adjustment be derived from the GI process of the \OTIS{} software?  And if so, can it offer any value to \MOM{}?
\end{itemize}



% SAL
Within the global ocean tide modelling literature, representation of self attraction and loading (SAL) has been identified as a significant and challenging detail since the mid-1970s \cite[pp189]{Cartwright:2000tt}.
Methods have been developed to deal with the complication of the recursive dynamics of ocean mass distribution and earth elasticity.   For the sake of computational simplicity, models such as \MOM{} have employed the simple scalar parameterisation as a modification to basic body forcing.\\
The scalar parameterisation appears to be an unnecessarily blunt compromise given the current availability of pre-computed global tide solutions.  Pre-computation offers more realistic representations of SAL \citep{Egbert:2002ug}.   But for an \OGCM{} this proposition raises questions of technical implementation.  Moreover, whilst a proper treatment of SAL is understood to be very important for global tides, the relevance to \OGCM{}s or any regional ocean model is not well-established.  



% temporal 
Temporal representation of the \ATGP{} was discussed in Section \ref{S:ATGP}.   Particular technical issues arise in connection to the barotropic forcing in \MOM{}.  For instance, the manner in which split-explicit scheme is implemented in connection to surface flux fields is not amenable to direct temporal formulations of the \ATGP{}.  Despite the apparent barrier, applying arbitrary pre-computed 2-dimensional force fields within the barotropic solver may be required to address modifications under discussion.  \\
It is noted that this temporal issue is not only relevant to the \ATGP{} and SAL.  Barometric pressure forcing is formulated identically and implementation is a closely related problem to explicit tides.\\



% OTIS
A more speculative question related to body forcing is raised in connection to the `weak constraint' data assimilation process of \OTIS{}.  
The generalised inverse method in \OTIS{} effectively calculates an adjustment to the body forcing in order to account for sources of mis-representation in the tide model.  Within the use of \OTIS{} this is presented as an objective trade-off between imperfect dynamics and imperfect observations.  The proposal here is to extract the forcing adjustment from \OTIS{} and use it to adjust forcing in a separate ocean model; namely \MOM{}.  Details of grid discretisation and bathymetry are important considerations in establishing applicability.\\
Associated technical challenges will be addressed. For instance, the translation of quantities from tidal spectral space to discrete time-stepping.\\



\subsubsection{Sources}
Boundary conditions and forcing fields; \\
Observations for evaluation.   Primarily, tide gauges

This work will require technical proficiency with both the data assimilating tide mode \OTIS{} [Figure \ref{fig:OTIS_eg}] as well as the primitive equation ocean model \MOM{}.\\
Observational verification will be employed using both tide gauges and radar altimeter results.  It is proposed that the coastal altimetric research products from the French agency CTOH will be of value and registration for use of this data has been completed.
%\citep{Schiller:2007gk}
%\citep{Egbert:2002ug}
%\citep{Stepanov:2004up}


\subsubsection{Method outlook}
This study will leverage capability developed towards the study outlined in Section \ref{S:plan_OTIS}; including a regional configuration of \OTIS{} and \MOM{}.  Again, the possible alternative of employing \ROMS{} is flagged.\\



Firstly, basic preparation and comparison of several spatial fields will be carried out.   This will include relevant candidates for the  \ATGP{}, SAL and \OTIS{}-derived body forcings.   Characterisation with regard to relative magnitude and phase differences will provide an initial appreciation of possible significance.\\



Secondarily, a simplified `cheap' configuration of the \OGCM{} will be employed to investigate questions of the forcing implementation.   
In this regard two basic approaches are highlighted:
\begin{inparaenum}[(a)]
\item modification of the present harmonic formulation calculated `on the fly' within the barotropic loop;
\item application of arbitrary pre-calculated fields to the barotropic solver.
\end{inparaenum}


In light of the preceding steps, a carefully selected set of \OGCM{} simulations will be developed and evaluated against observations. 



Ultimately this study aims to contribute to understanding the significance, or otherwise, of tidal body forcing formulations and inform future operational developments. 




