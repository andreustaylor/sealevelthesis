\chapter{More on global and modelled tides}
\label{appendix:globalTides}
\section{Global tide model details}
The Laplace Tidal Equations (LTE) evolve a simple 3 dimensional ocean state consisting of horizontal mass transport and sea level perturbation. In time domain, the LTE can be written following the notation and discussion of \cite[pp185]{Egbert:2002ug}:
%----------------------
% LTE in time-domain
\begin{align}
    \label{E:LTE_momtm}
    \frac{\delta \mathbf{U} }{ \delta t} + f\vec{k} \times \mathbf{U} + gH\nabla \eta  + \mathbf{F} &= \mathbf{f_0} + gH \nabla \eta_{SAL} \\
    \label{E:LTE_cont}
    \frac{\delta \mathbf{\eta} }{\delta t} &= -\nabla.\mathbf{U} 
\end{align}
%----------------------
Where $\mathbf{U}$ is the depth integrated horizontal transport, $\eta$ is the sea level signal, $H \gg \eta$ is mean water column depth, $f=2\Omega\sin\theta$ is the Coriolis parameter and $g$ vertical gravitation.\\
The forcing terms on the right hand side of the momentum equation \label{E:LTE_momtm} require explanation.  
$\mathbf{f_0}$ denotes the astronomical body forcing taking into account earth tide effects, $\mathbf{f_0} = gH\nabla\eta_{eq}$ following Equation \ref{eq:VT}.  
An approximation for SAL is here written as a separate forcing term to reflect the suitability of using values pre-computed from existing global tide models.   
Alternatively $\eta_{SAL}$ can be put on the left hand side of the equation as a scaled version of $\eta$ - as is the case in \MOM{}.   
This scalar approximation is relatively inaccurate as discussed in section \ref{sec:basic_potential}.\\
The frictional dissipation term $F$ is a particular source of complexity, especially with regard to parameterisation and linearisation.
Compared to the dynamics represented within an \OGCM{}, these tidal hydrodynamics simulate a more `aggregated' psuedofluid in that the processes contained within parameterisations are have a higher degree of complexity - as per the schematic in Figure \ref{fig:ogcmScales}
\begin{quotation}
Forward global tide models are an ideal testing ground for the hydrodynamical cores of numerical ocean general circulation models, and for ideas about drag and dissipation. In contrast to data-constrained models, forward models cannot achieve accurate tidal elevations unless substantial parameterised drag is included in the abyss. Forward models thus point clearly to drag in the open ocean as a central control on tidal flow.\citep{Arbic:2004wz}
\end{quotation} 
Linearisation is important in the tidal context to facilitate transformation of the LTE into the frequency domain.   Given the tidal LTI framework, the ultimate solution of a tidal model is the evaluation of static admittances or constants.   One approach towards this end would be to integrate the LTE in the time-domain and subsequently reduce the output to a series of harmonic constants via conventional analysis.   The celerity of barotropic waves in the deep ocean is relatively fast and thus requires short time-steps.   Assuming linearity and the existence of a convergent solution in frequency-space, the LTE can be solved directly in spectral form with greater computational efficiency.   The efficiencies are especially important upon the application of data-assimilation methods that involve both a backward and forward iteration of the dynamics \cite[pp184]{Egbert:2002ug}.


Again following the notation of \cite[pp186]{Egbert:2002ug}, the LTE at a single tidal frequency $\omega$ can be written:
%----------------------
% LTE in freq-domain
\begin{align}
\label{E:LTE_momtm_w}
\mathbf{\Omega} \mathbf{U} + gH\nabla \eta &= f_u \\
\label{E:LTE_cont_w}
\nabla.\mathbf{U} + i\omega\eta &= f_\eta\\
\mbox{where   } \Omega             &=
\left[ \begin{array}{cc} 
      i\omega + \kappa & f \\ 
       -f              & i\omega + \kappa  
                        \end{array} \right]   \nonumber
\end{align}
%----------------------
Where $\kappa$ is a linearised approximation for the dissipative stress.   Frequency-space forcing terms $f_u, f_\eta$ are written in both equations for generality with regard to data assimilation (inversion).  This frequency-space approach is common to other data assimilative tidal models such as FES \cite[pp395]{Lyard:2006ir}.\\
Nonlinear terms can be incorporated but complicate the formulation of any inversion.


Whilst numerically solving the LTE is quite tractable and comparatively simple, significant uncertainties prevent a direct `free' or `forward' model from producing accurate forecasts.  Hence the importance of data assimilation or generalised inversion methods.  Solving the LTE is complicated by spare observations and ``\textit{$\dots$ the need for accurate open boundary conditions and bottom topography, the need for approximate parameterisations of dissipation in the tidal equations, solid earth effects, and the effects of ocean stratification on the barotropic tides, which may be difficult to account for without full 3D modelling of baroclinic tidal currents}''\citep[183]{Egbert:2002ug}.

A comparative description of data assimilation methods in the context of tide models is laid out by Egbert and Bennet \cite{Egbert:1996vr}.



Summaries that categorise the many global tide models on the basis of design choices, parameterisations and data assimilation methods are given in \cite{Ardalan:2008gs} and \cite{Matsumoto:2000tg}. \\

%   ??  Application of the LTE is not restricted to the barotropic surface tide.   The formulation can be extended to represent stationary internal tidal modes via use of equivalent-depths .\\
% Energy cascades.??\\


%-----------------%
\section{Implementation of tidal forcing in \OGCM{}s }
%\label{S:numerical_impl}
The motivations for explicitly representing tides within an \OGCM{} where introduced in section \ref{sec:mesoscaleOperational}
The astronomical tidal forcing can be conceptualised as a relatively smooth global surface field that varies in time. 
Subsequently, perhaps the he most significant design choice for numerical implementation regards how to formulated the time variation of this field.
At face value, the \emph{direct formulation} provides the most unmediated and accurate representation of $\eta_{eq}$. Given spherical harmonics $(n,m) \in (2,0) , (2,1) , (2,2)$, the full global tidal forcing can be pre-computed $c_{nm}(t)$ as 6 real valued time series and simple trigonometric functions of location.
This approach has the advantage of accurately representing the \ATGP{} and facilitates incorporation of best-practice pre-computed SAL corrections \citep{Egbert:2002ug}. 
Furthermore, this approach could offer consistent numerical formulation with other barotropic forcing fields.


On the other hand, a direct formulation of the broadband tidal forcing introduces some programatic inconveniences and risks. 
For instance, the input time series must be extracted from a numerical ephemerides at timings to suit each particular simulation.  This is essentially a programing complication that adds an upstream data dependence.
Furthermore, the process for converting ephemerides to forcing fields is relatively opaque and introduces additional needs for integration testing measure to mitigate the risk of non-fatal errors arising from handing upstream dependencies (e.g. incorrect reading of dates).

Direct (or ``full'') tidal forcing is apparently not as common in the ocean literature as in gravity related studies. One instance of forcing a global hydrodynamic model directly is described by \cite{Weis:2008ex} and more recently \cite{Logemann:2020} - notably in an non-operational settings.   These authors stress that ``\textit{tidal predictions produced by such a model [free running and formulated with the full tidal potential], even after the inclusion of the LSA terms, will still be far away from the precision of current tidal models using data assimilation techniques}''.

In contrast, a harmonic formulation of $c_{nm}(t)$ for the same range of spatial harmonics requires the specification of many constants for each \emph{temporal} harmonic to be included.  
Hundreds of harmonic components \citep[pp3]{Desai:2006wo} would be required to match the spectral content of an equivilant direct formulation.


In practice only a small number (often 8) of the most powerful spectral clusters are specified as `primary constituents'.  
Truncation is based on the relative power of tidal lines within the harmonic development and subsequently requires nodal adjustments (as per equation \ref{E:Axb}).

Programatically the harmonic approach facilities some attractive simplifications.  
After the specification of a small number of constants, the temporal variation can be cheaply calculated within the ocean model via only a few lines of code.   
For relatively short simulations ($\ll$ 1 year), the nodal corrections are reasonably treated as fixed.   This situation is amenable to de-bugging and has no external dependancies.
It is also a relevant consideration that assessment of tidal output is conventionally based on harmonic fits to these same primary constituents.

Arbic et al \citep{Arbic:2010us}, apply body forcing using the temporal harmonic representation. In what appears to be common approach the authors describe developing their implementation via a progressive addition of temporal harmonics: no tides, M2-only, 8 primary constituents.  Treating M2 separately facilitates interpretation and comparison to other literature; including theoretical developments of single frequencies and tidal atlases.


%MOM
The public distribution of \MOM{} includes a very simple module for representing tidal body forcing in a manner suitable for climate simulations\cite[pp263] {Griffies:2008vh}.
The eight most powerful constituents (frequency clusters) from $\eta_{eq}$, due to only spherical harmonics $(n,m) = (2,1) , (2,2)$, are written with fixed harmonic amplitude.  SAL is simply parameterised with the scalar approximation. 
Astronomical phase information is neglected altogether.
Schiller and Feidler \citep{Schiller:2007gk} worked around this gap with addition of a hard-coded astronomical argument offset valid for a particular epoch.

Tidal body forcing is written as a depth-independent horizontal term in the momentum equation evolved at the barotropic timestep.
This cheap barotropic code reflects the climate-simulation background of \MOM{}.  
By default, arbitrary forcing terms such as surface fluxes are held constant over iterations of the inner barotropic loop.  
Whilst this is a reasonable optimisation given relatively slow forcing timescales, it is not valid for $\eta_{eq}$ which varies powerfully and quickly.
By employing the temporal harmonic approach to representing the tidal force, $\eta_{eq}$ can be updated algebraically at the barotropic timestep, without requiring relatively expensive file input/output.


%-----------------%
\subsection{Complications for tidal sea level}
The impact of lateral boundaries and shallow water effects on representing global tides is a topic that arises in time-stepped forward models.
In a barotropic simulation forced directly by tidal ephemerides \cite{Weis:2008ex}, the authors indicate that solutions for \emph{deep-water} partial tides are significantly influenced by the explicit simulation of broad-band tidal spectrum.   
(It is notable that this simulation did not include a `wave drag' term - but the authors cite this exclusion as a likely source of error \citep[pp5]{Weis:2008ex})

Based on a more thorough and analytical approach, Arbic et al investigations provide a similar conclusion:
\noindent \begin{quotation}
$\dots{}$ the back-effect of coastal tides upon open-ocean tides is demonstrated in numerical experiments in which removal of regions of resonant coastal tides significantly alters tidal amplitudes (generally, increasing them) and phases, over basin-wide and even global scales.\citep[pp263]{Arbic:2009in}
\end{quotation}

With regard to operational \OGCM{}s these discussions are taken to highlight the potential impact of lateral conditions designed for nontidal simulations.  
One instance are the so-called `earth-works', where bathymetry and coastlines are manually adjusted in the interest of allowing certain ocean circulation features to exist.  
Similarly relevant is the representation of barotropic dissipation in shelf regions. 
Specific cases that may have an impact on the Australian coastline include the parameterisation of bottom dissipation over the Great Barrier Reef, and possibly the geometry of coastline features such as the Gulf of St Vincent and King Sound.


Inclusion of explicit tides within an \OGCM{}  offers the potential to improve the simulation of the global ocean state, but in doing so introduces many novel challenges.

How to approach data assimilation is particularly problematic.  Assimilation of corrected observations and the exclusion of tidal dynamics provides has been more or less fundamental to the design of the current generation of \GODAE{} systems to date.
Sea level observations provide a powerful constraint upon operational ocean models, and how to assimilate sea level into models that dynamically include both mesoscale and tidal motions is an open question.   
